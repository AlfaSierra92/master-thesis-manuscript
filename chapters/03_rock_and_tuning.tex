\chapter{Testbed}

Per l'implementazione del testbed necessario a eseguire una varietà di test, sono state esaminate numerose schede di sviluppo disponibili sul mercato, caratterizzate da prezzi competitivi e disponibilità immediata. A causa dell'aumento esponenziale dei costi delle varie versioni di Raspberry Pi, soprattutto a seguito della pandemia di Covid-19 e della conseguente scarsità di offerta, si è deciso di escludere a priori queste opzioni.

Inizialmente, si è considerata l'Arduino Yun, una scheda del noto brand Arduino. Tuttavia, si è subito riscontrato che le sue prestazioni erano insufficienti e che gli strumenti forniti non garantivano la flessibilità desiderata. In particolare, la shell risultava poco performante e non era possibile impostare tempi di sleep inferiori a un secondo. Inoltre, è importante notare che l'Arduino Yun è stato deprecato a favore di altre schede, che, sebbene più performanti, non offrivano lo stesso rapporto qualità-prezzo.

Successivamente, si è passati all'analisi di router Mikrotik, ma anche in questo caso le prestazioni e la flessibilità nelle configurazioni non hanno soddisfatto le aspettative.

Alla fine, la soluzione ottimale è stata trovata nelle schede Rock 3 Model A. Queste schede, pur essendo leggermente più economiche rispetto ai Raspberry Pi, offrivano le prestazioni, la flessibilità e il costo contenuto ricercati. Sono state utilizzate quattro unità, tutte configurate in modo identico con Raspbian OS come sistema operativo. Ulteriori dettagli verranno forniti in seguito.

\section{Rock 3 Model A board}

Il Rock 3 Model A è una scheda di sviluppo avanzata, progettata per applicazioni che richiedono elevate prestazioni di calcolo, rendendola particolarmente adatta per progetti di Internet of Things (IoT), edge computing e server leggeri. 

Al centro di questa scheda si trova il processore Rockchip RK3566, un potente quad-core ARM Cortex-A55 che può raggiungere una frequenza di clock fino a 2.0 GHz. Questa architettura a 64 bit consente di gestire una vasta gamma di applicazioni moderne in modo efficiente. La scheda è dotata di 2 GB di RAM LPDDR4, che garantiscono prestazioni elevate e una gestione efficace delle applicazioni multitasking. Per quanto riguarda l'archiviazione, il Rock 3 Model A offre uno slot per schede microSD, permettendo di espandere facilmente la capacità di memoria, oltre a supportare eMMC fino a 64 GB. Per la connettività, è equipaggiata con una porta Gigabit Ethernet, che assicura una connessione di rete ad alta velocità, e diverse porte USB 3.0 e USB 2.0 per il collegamento di dispositivi esterni. 

Sebbene non sia una caratteristica fondamentale per i nostri scopi, è interessante notare che la scheda include anche una GPU Mali-G52, che consente una buona elaborazione grafica, rendendola adatta per applicazioni multimediali e giochi leggeri. Il Rock 3 Model A offre ampie possibilità di espandibilità, grazie ai pin GPIO (General Purpose Input/Output) che consentono di collegare sensori, attuatori e altri dispositivi. Supporta varie interfacce di comunicazione, come I2C, SPI e UART, facilitando l'integrazione con altri componenti hardware. Per quanto riguarda l'alimentazione, la scheda può essere alimentata tramite un connettore DC o tramite USB-C, offrendo flessibilità nelle opzioni di alimentazione.

La compatibilità con diversi sistemi operativi, tra cui varie distribuzioni Linux come Raspbian e Android, rende il Rock 3 Model A estremamente versatile per una varietà di progetti; in particolare, nel nostro caso, si è preferito adottare Raspbian OS in quanto risulta essere quello più simile ad un classico sistema \textit{Linux-based}.

\subsection[Atheros Wi-Fi card]{Atheros Wi-Fi card}
Il Rock 3 descritto sopra non monta \textit{out-of-the-box} una scheda di rete che fornisce la connettività Wireless (IEEE 802.11), in virtù di ciò si è reso necessario aggiungerne una esterna collegandola mediante l'interfaccia \textit{PCI Express - M.2 Specification} che la board fornisce per permettere l'espansione delle sue funzionalità hardware.

Tra le varie soluzione in commercio, la scelta è ricaduta sull'adattatore Qualcomm Atheros AR9462. Esso supporta gli standard Wi-Fi 802.11a/b/g/n e permette l'utilizzo della modalità OCB (\textit{Outside the Context of BSS}) utilizzata nel nostro caso al fine di permettere la comunicazione senza fili tra i vari dispositivi senza la necessità di stabilire un \textit{Basic Service Set}.

Infine, la natura completamente open-source dei suoi driver ha permesso di apportare ad esso le modifiche necessarie al nostro scopo.

\subsection[Connessione con le board]{Connessione con le board}
Le quattro schede utilizzate nel progetto, come descritto prima, sono dotate di due interfacce di rete distinte, ciascuna con un ruolo specifico nel funzionamento complessivo del sistema. La prima interfaccia è una connessione wireless, utilizzata esclusivamente per l'esecuzione dei test.

La seconda interfaccia è una connessione Ethernet, progettata specificamente per consentire un'interazione diretta con le schede, permettendo così l'invio di comandi per la configurazione, l'avvio e la chiusura dei test, tutti gestiti da un PC.

I quattro dispositivi sono stati collegati a uno switch Ethernet a cinque porte, mentre la porta aggiuntiva è stata riservata per la connessione del PC. Nella seguente immagine è mostrata la topologia di rete con i vari indirizzi IP assegnati.

La connessione avviene mediante il protocollo sicuro SSH, ma non ci si soffermerà su di esso in quanto questo esula dagli obiettivi del presente testo.

\section{Tweaking}
Per rendere possibili alcune configurazioni nei dispositivi Rock e la raccolta dei dati, si è reso necessario mettere mano nalel codice sorgente del Kernel Linux ed applicare alcune patch.

Nella fattispecie, si è dovuto provvedere ad abilitare la \textit{Debug Mode} messa a disposizione dal driver \textit{Ath9k} della scheda di rete wireless ma disabilitata in maniera predefinita, una piccola modifica all'output delle statistiche di rete accessibili dall'utente e l'abilitazione delle code separate per le quattro \textit{Access Categories}.

Tutto ciò è stato reso possibile grazie a patch e a informazioni fornite dalla comunità open source e alla documentazione del produttore.

Viene fornita, inizialmente, una descrizione separata delle varie modifiche apportate, esplicitiando alla fine del trittico il processo di compilazione necessario. Importante notare che le varie patch sono state scritte esclusivamente per la versione del kernel Linux \textit{rockchip64-6.6}, l'ultima disponibile al momento della configurazione del testbed; non si garantisce che possano essere applicate senza ulteriori modifiche ad altre versioni, sia che esse siano sulla architettura ARM che differente.

\subsection[Debug Mode]{Debug Mode}
Per l'abilitazione della suddetta modalità, si è semplicemente messo mano al file \textit{linux-rockchip64-current.config}, file di configurazione utilizzato dalla \textit{toolchain} di compilazione del Kernel del sistema Armbian, abilitando tre differenti flag\cite{linux_wireless}:

\verb|CONFIG_ATH9K_HTC_DEBUGFS=y|

\verb|CONFIG_ATH9K_HWRNG=y|

\verb|CONFIG_ATH9K_DEBUGFS=y|

\subsection[Queues]{Queues}
La patch fornita modifica un segmento di codice nel file wme.c del driver mac80211, che gestisce la classificazione dei pacchetti in una rete wireless.
Molto semplicemente, si è provveduto ad eliminare la classificazione effettuata dal driver \textit{mac80211} in quanto il suo comportamento andava in conflitto con le Access Categories utilizzate e dirottava tutti i pacchetti sulla coda \textit{Best Effort (BE)}, indipendentemente dalla categoria scelta.
Il problema è dovuto all'introduzione delle cosiddette "code software intermedie" all'interno del driver "mac80211", per i driver supportati\cite{intermediate_queue} come l'\textit{Ath9k}.

\begin{lstlisting}
    --- a/net/mac80211/wme.c        2024-06-17 11:16:29
    +++ b/net/mac80211/wme.c        2024-06-17 11:17:12
    @@ -176,9 +176,9 @@
     
            /* use the data classifier to determine what 802.1d tag the
             * data frame has */
    -       qos_map = rcu_dereference(sdata->qos_map);
    -       skb->priority = cfg80211_classify8021d(skb, qos_map ?
    -                                              &qos_map->qos_map : NULL);
    +       //qos_map = rcu_dereference(sdata->qos_map);
    +       //skb->priority = cfg80211_classify8021d(skb, qos_map ?
    +                                              //&qos_map->qos_map : NULL);
     
      downgrade:
            return ieee80211_downgrade_queue(sdata, sta, skb);
\end{lstlisting}
Questa caratteristica è stata introdotta per spostare l'implementazione delle code più verso il lato software del sottosistema wireless, consentendo all'hardware di mantenere solo code brevi e permettendo anche una maggiore equità tra le stazioni che comunicano.

La possibilità di utilizzare correttamente le categorie di accesso insieme alle code software del \textit{mac80211} risulta essere in fase di implementazione da parte della comunità al momento della stesura di questo elaborato.

\subsection[Statistiche occupazione canale]{Statistiche occupazione canale}
La patch fornita modifica un segmento del codice nel file link.c del driver \textit{ath9k}, che è parte della gestione delle statistiche di monitoraggio della rete wireless. Esso in maniera predefinita restituiva, mediante il comando \verb|iw wlp1s0 survey|, varie statistiche tra cui i tempi in cui, rispettivamente, il canale risulta essere occupato, viene usato per trasmettere e ricevere da parte dell'interfaccia. 

Questi valori venivano mostrati in output in maniera incrementale, ovvero ogni valore campionato era sommato al valore precedente; tale modus operandi complicava la raccolta dei dati nel nostro caso e si è provveduto a modificare il codice in modo che i valori vengano mostrati, singolarmente, volta per volta.
\begin{lstlisting}
    --- a/drivers/net/wireless/ath/ath9k/link.c	2024-06-21 11:46:11
    +++ b/drivers/net/wireless/ath/ath9k/link.c	2024-06-21 11:47:38
    @@ -524,10 +524,10 @@
                 SURVEY_INFO_TIME_BUSY |
                 SURVEY_INFO_TIME_RX |
                 SURVEY_INFO_TIME_TX;
    -		survey->time += cc->cycles / div;
    -		survey->time_busy += cc->rx_busy / div;
    -		survey->time_rx += cc->rx_frame / div;
    -		survey->time_tx += cc->tx_frame / div;
    +		survey->time = cc->cycles / div;
    +		survey->time_busy = cc->rx_busy / div;
    +		survey->time_rx = cc->rx_frame / div;
    +		survey->time_tx = cc->tx_frame / div;
         }
     
         if (cc->cycles < div)
\end{lstlisting}
Giusto per completezza, viene mostrato l'output dell'esecuzione del comando 

\verb|iw| \verb|wlp1s0 survey dump|:
\begin{lstlisting}
root@rock-3a:~# iw wlp1s0 survey dump
    Survey data from wlp1s0
	frequency:			2462 MHz [in use]
	noise:				-93 dBm
	channel active time:		136546 ms
	channel busy time:		1642 ms
	channel receive time:		1496 ms
	channel transmit time:		0 ms
\end{lstlisting}
Per semplicità viene mostrato qui solo l'output relativo al canale 11, omettendo le informazioni relative agli altri canali.
\subsection[Kernel building]{Kernel building}

\subsection[Impostazione interfaccia Wi-Fi]{Impostazione interfaccia Wi-Fi}

\section{iPerf con Access Categories abilitate}

\subsection[Editing del sorgente]{Editing del sorgente}

\subsection[Compilazione]{Compilazione}

\section{Netcat OpenBSD}