\chapter{Appendice}

\section{Patch iPerf per AC}
\label{iperf_ac}
\begin{lstlisting}
    --- a/include/Settings.hpp
    +++ b/include/Settings.hpp
    @@ -135,6 +135,7 @@ typedef struct thread_Settings {
         // int's
         int mThreads;                   // -P
         int mTOS;                       // -S
    +    int mMACUP;                     // -A
     #if WIN32
         SOCKET mSock;
     #else
    --- a/include/version.h
    +++ b/include/version.h
    @@ -1,4 +1,4 @@
    -#define IPERF_VERSION "2.0.13"
    +#define IPERF_VERSION "2.0.13 OpenWrt-V2X patch"
     #define IPERF_VERSION_DATE "21 Jan 2019"
     #define IPERF_VERSION_MAJORHEX 0x00020000
     #define IPERF_VERSION_MINORHEX 0x000D0003
    --- a/src/Locale.c
    +++ b/src/Locale.c
    @@ -103,6 +103,7 @@ Server specific:\n\
       -s, --server             run in server mode\n\
       -t, --time      #        time in seconds to listen for new connections as well as to receive traffic (default not set)\n\
           --udp-histogram #,#  enable UDP latency histogram(s) with bin width and count, e.g. 1,1000=1(ms),1000(bins)\n\
    +  -A, --accesscategory <AC> Forces a certain EDCA MAC access category to be used (BK, BE, VI, VO)\n\
       -B, --bind <ip>[%<dev>]  bind to multicast address and optional device\n\
       -H, --ssm-host <ip>      set the SSM source, use with -B for (S,G) \n\
       -U, --single_udp         run in single threaded UDP mode\n\
    @@ -125,6 +126,7 @@ Client specific:\n\
     "  -n, --num       #[kmgKMG]    number of bytes to transmit (instead of -t)\n\
       -r, --tradeoff           Do a bidirectional test individually\n\
       -t, --time      #        time in seconds to transmit for (default 10 secs)\n\
    +  -A, --accesscategory <AC> Forces a certain EDCA MAC access category to be used (BK, BE, VI, VO)\n\
       -B, --bind [<ip> | <ip:port>] bind ip (and optional port) from which to source traffic\n\
       -F, --fileinput <name>   input the data to be transmitted from a file\n\
       -I, --stdin              input the data to be transmitted from stdin\n\
    --- a/src/PerfSocket.cpp
    +++ b/src/PerfSocket.cpp
    @@ -155,6 +155,15 @@ void SetSocketOptions( thread_Settings *
         }
     #endif
     
    +   	// set MAC AC (access category) field, if specified only (i.e. if mMACUP != -1)
    +    // AC is set starting from user priorities (UP)
    +	if ( inSettings->mMACUP >= 0 ) {
    +		int  up = inSettings->mMACUP;
    +		Socklen_t len = sizeof(up);
    +		int rc = setsockopt( inSettings->mSock, SOL_SOCKET, SO_PRIORITY, (char*) &up, len );
    +		WARN_errno( rc == SOCKET_ERROR, "setsockopt SO_PRIORITY" );
    +	}
    +
         if ( !isUDP( inSettings ) ) {
             // set the TCP maximum segment size
             setsock_tcp_mss( inSettings->mSock, inSettings->mMSS );
    --- a/src/Settings.cpp
    +++ b/src/Settings.cpp
    @@ -131,6 +131,7 @@ const struct option long_options[] =
     {"realtime",         no_argument, NULL, 'z'},
     
     // more esoteric options
    +{"accesscategory",  required_argument, NULL, 'A'},
     {"bind",       required_argument, NULL, 'B'},
     {"compatibility",    no_argument, NULL, 'C'},
     {"daemon",           no_argument, NULL, 'D'},
    @@ -198,6 +199,7 @@ const struct option env_options[] =
     {"IPERF_REPORTSTYLE",required_argument, NULL, 'y'},
     
     // more esoteric options
    +{"IPERF_MACAC",        required_argument, NULL, 'A'},
     {"IPERF_BIND",       required_argument, NULL, 'B'},
     {"IPERF_COMPAT",           no_argument, NULL, 'C'},
     {"IPERF_DAEMON",           no_argument, NULL, 'D'},
    @@ -218,7 +220,8 @@ const struct option env_options[] =
     
     #define SHORT_OPTIONS()
     
    -const char short_options[] = "1b:c:def:hi:l:mn:o:p:rst:uvw:x:y:zB:CDF:H:IL:M:NP:RS:T:UVWXZ:";
    +// Edited to add the A: (A + 1 argument) short option
    +const char short_options[] = "1b:c:def:hi:l:mn:o:p:rst:uvw:x:y:zA:B:CDF:IL:M:NP:RS:T:UVWZ:";
     
     /* ------------------------------------------------------
      * defaults
    @@ -279,6 +282,7 @@ void Settings_Initialize( thread_Setting
         //main->mThreads      = 0;           // -P,
         //main->mRemoveService = false;      // -R,
         //main->mTOS          = 0;           // -S,  ie. don't set type of service
    +    main->mMACUP        = -1;            // -A (set to an invalid number as default -> with -1 no setsockopt will be called for AC)
         main->mTTL          = -1;            // -T,  link-local TTL
         //main->mDomain     = kMode_IPv4;    // -V,
         //main->mSuggestWin = false;         // -W,  Suggest the window size.
    @@ -692,6 +696,24 @@ void Settings_Interpret( char option, co
                 mExtSettings->mTOS = strtol( optarg, NULL, 0 );
                 break;
     
    +        case 'A': // 802.11p/802.11e MAC layer access categories
    +            // Mapping between UP (0 to 7) and AC (BK to VO)
    +            if(strcmp(optarg,"BK") == 0) {
    +                mExtSettings->mMACUP=1; // UP=1 (2) is AC_BK
    +            } else if(strcmp(optarg,"BE") == 0) {
    +                mExtSettings->mMACUP=0; // UP=0 (3) is AC_BE
    +            } else if(strcmp(optarg,"VI") == 0) {
    +                mExtSettings->mMACUP=4; // UP=4 (5) is AC_VI
    +            } else if(strcmp(optarg,"VO") == 0) {
    +                mExtSettings->mMACUP=6; // UP=6 (7) is AC_VO
    +            } else {
    +                // Leave to default (-1), i.e. no AC is set to socket, and print error
    +                fprintf(stderr, "Invalid AC specified with -A\nValid ones are: BK, BE, VI, VO\nNo AC will be set\n");
    +            }
    +
    +            //mExtSettings->mMACUP = strtol( optarg, NULL, 0 );
    +            break;
    +
             case 'T': // time-to-live for both unicast and multicast
                 mExtSettings->mTTL = atoi( optarg );
                 break;
\end{lstlisting}