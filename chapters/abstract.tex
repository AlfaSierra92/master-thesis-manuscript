\thispagestyle{plain}
\section*{Abstract}
Negli ultimi due decenni, la sicurezza stradale ha beneficiato di significativi progressi tecnologici, in particolare attraverso lo sviluppo di Intelligent Transport Systems (ITS) e Vehicular Ad-hoc Networks (VANET). Questo lavoro esplora l'importanza di tali sistemi nel migliorare la comunicazione tra veicoli (V2V) e tra veicoli e infrastrutture (V2I), evidenziando il ruolo cruciale della standardizzazione dei Cooperative Awareness Messages (CAMs) e dei Decentralized Environmental Notification Messages (DENMs). L'implementazione di tecnologie come la Dedicated Short-Range Communication (DSRC) ha facilitato lo scambio di informazioni essenziali per applicazioni di Collision Avoidance (CA), contribuendo a ridurre gli incidenti stradali.

Il focus principale di questo studio è l'analisi delle politiche di Quality of Service (QoS) in ambienti Linux per dispositivi che supportano lo standard IEEE 802.11p, un aspetto fondamentale per il Decentralized Congestion Control (DCC). Attraverso un testbed composto da dispositivi Rock, si esamineranno le performance dei protocolli IEEE 802.11p e DCC, considerando vari parametri di prestazione. La gestione della congestione si rivela cruciale per garantire l'efficacia delle applicazioni di sicurezza e infotainment nelle VANET. Utilizzando strumenti come iPerf, si valuteranno throughput e variabilità delle prestazioni della rete, fornendo dati essenziali per ottimizzare l'esperienza utente in scenari critici. Questo lavoro intende contribuire alla ricerca nel campo della mobilità intelligente, promuovendo soluzioni innovative per una guida più sicura ed efficiente.
