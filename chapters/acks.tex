\chapter*{Ringraziamenti} % * per non numerare il capitolo
Vorrei chiudere questo capitolo della mia vita con una citazione, rischiando di cadere nell'ovvio, ma sicuro del suo valore. Albert Einstein disse: "\textit{Tutti sono dei geni, ma se giudichi un pesce dalla sua capacità di arrampicarsi sugli alberi, passerà la vita a credersi stupido}". Da oggi, mi auguro di smettere di arrampicarmi sugli alberi e, invece, di tuffarmi in mare. Chissà, potrei scoprire di essere un ottimo nuotatore.

Ora, arrivati ai titoli di coda di questa opera "biblica", è arrivato il momento di ringrazire chi c'è stato e chi mi è stato vicino anche da lontano, sicuro che tutti hanno fatto una parte, anche se piccola, nel proseguimento del mio percorso universitario ed anche di vita.

Il primo ringraziamento sento di darlo a me stesso, il quale una primavera di due anni fa, guardandosi indietro, decise di intraprendere un viaggio non del tutto inaspettato per raggiungere nuove vette. Ne è valsa la pena? Mi basta vedere la vista che c'è da qui.

Mi sento in dovere di ringraziare  il \textbf{Prof. Carlo Augusto Grazia}, mio docente ed anche relatore di questa tesi, per la sua professionalità, per la disponibilità che ha concesso me e il supporto datomi in questo lavoro conclusivo del mio percorso universitario magistrale. 

Non posso non spendere due parole per ringraziare mia madre e il mio amico Anthony: anche se non siete qui sono sicuro che sarete fieri di me.

Un ringraziamento va a mio padre, per la fiducia e l'accettazione della mia scelta nonostante la lunga distanza.

Ringrazio i miei cugini Yaya e Pasquale, senza dimenticare la neoarrivata Adele, per i "primi giorni" qui, per i loro consigli e il loro ininterrotto supporto. Ringrazio mia zia Maria e mio zio Gerardo per il loro aiuto, nonostante la distanza. Grazie anche ad Annalisa, Francesca, Alessio, Giulia, Anita e Roberto, qui c'è anche del vostro.

Iniziando i ringraziamenti agli amici, il primo non può che andare a Beatrice: per le lunghe chiacchierate, per la sua amicizia e la fiducia che ha riposto in me chiedendomi consiglio, e per avermi spronato nei momenti di bisogno, specialmente alla vigilia degli esami. Un grazie sincero va anche a voi, cari colleghi, per la vostra amicizia (sono o non sono l'unico e inimitabile Totò?): il compare Davide, compagno di tanti progetti, Diego, Luca “\textit{Palermo}”, Luca “\textit{Abruzzo}”, Marco e Francesco. Un pensiero speciale va anche agli amici di Modena, vecchi e nuovi: Michela, Felicia, Daniel (sì, la tesi l'ho finalmente finita!), Angela, e ovviamente ai miei coinquilini con cui ho condiviso questi anni di vita. Grazie anche agli amici di giù, con cui ho trascorso le poche ferie in Salento. E un ringraziamento speciale va a chi mi ha spronato a “\textit{lasciare la panchina e scendere in campo}” facendomi capire che, invece di continuare ad arrampicarmi, avrei dovuto imparare a nuotare. Senza di voi, alcune scalate sarebbero state davvero troppo difficili da affrontare.

\noindent "\textit{I am glad you are here with me. Here at the end of all things, Sam.}"

\begin{flushright}
    \textbf{\textit{Antonio}}
\end{flushright}