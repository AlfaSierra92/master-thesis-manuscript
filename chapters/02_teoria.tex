\chapter{Cenni teorici}

\section{ITS - Intelligent Transport System}

\section{VANET}

\section{IEEE 802.11p (WAVE)}
Quando si introducuno il concetto di VANET e i relativi protocolli utilizzati, è comune riferirsci al \textit{IEEE 902.11p} anche chiamato WAVE.

WAVE, acronimo di Wireless Access in Vehicular Environments, è una tecnologia progettata per migliorare la comunicazione nei contesti di trasporto. Per supportare questa tecnologia, l'IEEE ha sviluppato una serie di standard, già menzionati nel capitolo 1, che includono l'emendamento 802.11p al protocollo 802.11, introdotto nel 2010 e aggiornato nel 2016, e la famiglia di standard 1609.x.

In sintesi, WAVE si occupa sia del livello fisico (PHY) che del livello MAC, utilizzando l'802.11p come livello fisico. L'architettura del protocollo WAVE è rappresentata nella figura 2.1.

Come illustrato nella figura, lo stack WAVE supporta sia protocolli IP che non IP a livelli superiori. In particolare, i frame non-IP vengono trasmessi tramite un protocollo dedicato chiamato WAVE Short Message Protocol (WSMP), specificato nel documento IEEE 1609.3. Questo protocollo è progettato per minimizzare l'overhead delle comunicazioni, consentendo così una gestione più efficiente delle esigenze critiche delle reti veicolari.

WSMP offre funzioni di rete e di trasporto attraverso due intestazioni denominate WSMP-N-Header e WSMP-T-Header. I messaggi che utilizzano il WSMP sono noti come WAVE Short Messages, o semplicemente WSM [27].


\subsection[Physical layer]{Physical layer}

\subsection[MAC layer]{MAC layer}

\subsection[EDCA]{EDCA}

\subsection[IEEE 1609.4 for multi-channel operations]{IEEE 1609.4 for multi-channel operations}