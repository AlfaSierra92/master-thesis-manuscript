\chapter{Conclusioni e sviluppi futuri}

%\section{due}
E' stata condotta un'analisi approfondita dell'impatto della congestione sul canale di trasmissione in contesti di comunicazione wireless, in particolare focalizzandosi sui dispositivi trasmittenti che operano secondo lo standard IEEE 802.11p. Questo standard, progettato per le comunicazioni nelle reti di veicoli (VANET), offre notevoli vantaggi in termini di latenza e affidabilità, ma si trova ad affrontare sfide significative quando si tratta di gestire la congestione del canale. I risultati ottenuti hanno dimostrato chiaramente che la congestione può compromettere seriamente il throughput delle trasmissioni, causando rallentamenti e aumentando i tempi di latenza, fattori che possono influenzare negativamente l'efficacia delle applicazioni in tempo reale, come i sistemi di navigazione, i servizi di sicurezza stradale e la guida remota di veicolari, sia per trasporto persone e non.

Nel corso della ricerca, sono state esaminate differenti casistiche. È emerso che, in assenza di meccanismi adeguati di gestione della congestione, il numero crescente di dispositivi connessi può portare a una saturazione del canale, riducendo drasticamente la qualità del servizio. Questo fenomeno è particolarmente critico in scenari urbani, dove il traffico di veicoli e dispositivi è elevato e le comunicazioni devono avvenire in tempo reale per garantire la sicurezza e l'efficienza del sistema di trasporto.

Una delle soluzioni più promettenti emerse dall'analisi è l'implementazione di classi di priorità per i messaggi trasmessi. L'assegnazione di priorità diverse ai vari tipi di comunicazione consente di gestire in modo più efficace le risorse di rete, garantendo che i messaggi più critici ricevano la banda necessaria per una trasmissione fluida. In particolare, anche solo l’adozione di una componente dell’algoritmo \textit{Decentralized Congestion Control (DCC)}, ovvero le singole code di priorità, ha dimostrato di poter migliorare notevolmente la situazione, permettendo di ottimizzare l'uso del canale di trasmissione e di aumentare il throughput complessivo.

I risultati sperimentali hanno evidenziato che, quando vengono applicate classi di priorità più elevate, la disponibilità di banda per tali trasmissioni aumenta in modo significativo. Questo non solo migliora l'efficienza delle comunicazioni, ma consente anche di ridurre i tempi di attesa per i messaggi critici, contribuendo a un sistema di comunicazione più reattivo e affidabile.

Le implicazioni pratiche di queste scoperte sono rilevanti per lo sviluppo di sistemi di comunicazione più robusti e reattivi, in grado di supportare le crescenti esigenze delle applicazioni veicolari e di trasporto intelligente.

In conclusione, i risultati di questa ricerca offrono un contributo significativo alla comprensione e alla gestione della congestione nei sistemi IEEE 802.11p. Sebbene siano stati compiuti progressi notevoli, è fondamentale continuare a esplorare e sviluppare strategie innovative per affrontare le sfide future. Ulteriori studi potrebbero concentrarsi sull'integrazione di queste tecniche in scenari reali, valutando l'efficacia delle soluzioni proposte in condizioni operative variabili e complesse. Un interessante spunto per future ricerche potrebbe essere quello di valutare l'implementazione di \textit{datarate} più elevati rispetto a quelli attualmente consentiti dallo standard, anche mediante l'esplorazione di modulazioni numeriche più efficienti rispetto a quelle attualmente in uso, con l'obiettivo di massimizzare la capacità di trasmissione senza compromettere l'affidabilità delle comunicazioni. Questo approccio non solo contribuirà a un miglioramento continuo delle performance di rete, ma potrà anche fornire spunti preziosi per la progettazione di standard futuri, in grado di rispondere alle esigenze di un ambiente di mobilità sempre più interconnesso e dinamico.