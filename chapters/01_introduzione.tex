\chapter{Introduzione}

\section{Intelligent Transport System}
Fin dai primi anni del XX secolo, con la nascita dell'industria automobilistica, i produttori di automobili hanno cercato di migliorare i sistemi di sicurezza installati all'interno dei veicoli, ovviamente tutto in funzione di quanto la tecnologia potesse offrire al momento. Si è partiti, ad esempio, dall'introduzione dei freni idraulici su tutte le quattro ruote, per poi passare alle cinture di sicurezza, alla disponibilità degli airbag e all'introduzione dei primi sistemi quali ABS e simili. Ogni nuova tecnologia ha richiesto anni di ricerca e test prima di essere rilasciata e diventare uno standard per i nuovi modelli.

Negli ultimi ventennio, con l'obiettivo di aumentare la sicurezza dei veicoli, sono stati compiuti notevoli sforzi nel campo degli \textit{Intelligent Transport System (ITS)}, anche alla luce di quanto definito dalla direttiva 2010/40/EU dell'Unione Europea \cite{2010-40}. Questo paradigma comprende una serie di applicazioni e servizi utili non solo per la sicurezza del conducente e dei passeggeri, ma anche per la loro esperienza di guida, per l'efficienza del traffico e diverse altre esigenze di trasporto. Le applicazioni legate alla sicurezza e all'efficienza sono interconnesse e si avvantaggiano reciprocamente, motivo per cui l'industria automobilistica, enti governativi e numerosi ricercatori accademici collaborano per standardizzare tutti gli aspetti dei sistemi ITS.

I progressi verso la realizzazione dei sistemi ITS sono stati alimentati da importanti sviluppi nelle \textit{Vehicular Ad-hoc NETworks (VANET)}. Questa tecnologia è stata fondamentale per il successo dei sistemi ITS, consentendo uno scambio rapido e diretto di informazioni necessarie per la maggior parte delle applicazioni ITS. L'introduzione delle VANET, attraverso l'uso della \textit{Dedicated Short-Range Communication (DSRC)}, ha reso possibile lo scambio di messaggi tra Veicolo e Veicolo (V2V) e tra Veicolo e Infrastruttura (V2I).

Uno dei punti cardine, nell'implementazione delle comunicazioni V2V e V2I, è stato quello della standardizzazione, da parte di ETSI, dei cosiddetti \textit{Cooperative Awareness Messages (CAM)} che trova la sua applicazione principale nelle applicazioni deputate alla \textit{Collision Avoidance}, ovvero tutti gli strumenti applicativi atti a prevenire tutta una serie di situazioni di pericolo tra i vari veicoli e tutto ciò che può essere presente sulla sede stradale.

Le CA beneficiano dell'aumento del numero di sensori e della potenza di calcolo presenti nei veicoli moderni, che hanno già portato a funzionalità innovative come la Frenata Automatica d'Emergenza. Tuttavia, queste capacità sono limitate a una visione locale del veicolo; l'idea è di ampliare questa visione condividendo informazioni tramite le VANET con altri veicoli e unità stradali, con l'obiettivo di abilitare applicazioni più complesse ed efficaci. 

Questo implicitamente porta a creare una rete composta da nodi, ovvero veicoli e infrastrutture a bordo strada, che porta a dover affrontare svariati problemi sia implementativi che di performance, i quali richiedono particolari accorgimenti differenti da quelli presi nelle reti tradizionali, in quanto le condizioni di operatività sono completamente diverse.

\section{VANET}
Una \textit{VANET (Vehicular Ad-hoc Network)} è una classe distintiva di \textit{MANET (Mobile Ad-hoc Network)} in cui i veicoli in movimento fungono da nodi o router per scambiare messaggi tra di loro, o come \textit{Access Point (AP)}. Solitamente, una VANET può connettere veicoli entro un raggio di 100-900 metri utilizzando lo standard 802.11p. Il suo obiettivo è supportare sia la comunicazione Veicolo a Veicolo (V2V) che Veicolo a Infrastruttura (V2I) in una rete senza infrastruttura. Sono state avviate numerose iniziative di ricerca, come COOPERS, CVIS, SAFESPOT, PReVENT, Wireless Access in Vehicular Environments (WAVE) e Advanced Safety Vehicle Program (ASV) in Europa, negli Stati Uniti e in Giappone, per rendere gli ITS una realtà. 

Le VANET vengono utilizzate per supportare applicazioni critiche per la sicurezza e applicazioni di intrattenimento non legate alla sicurezza. Le applicazioni di sicurezza, come l'evitamento delle collisioni, la rilevazione pre-collisione o il cambio di corsia, mirano a ridurre gli incidenti stradali attraverso il monitoraggio e la gestione del traffico. Le applicazioni non di sicurezza consentono ai passeggeri di accedere a vari servizi come internet, comunicazione interattiva, giochi online, servizi di pagamento e aggiornamenti informativi mentre i veicoli sono in movimento. La principale differenza tra le applicazioni di sicurezza e quelle non di sicurezza è che le prime sono in grado di inviare e elaborare messaggi in tempo reale. Sia i conducenti che i passeggeri possono accedere a entrambi i tipi di servizi dall'infrastruttura vicina in modo fluido utilizzando tecnologie di accesso wireless.

Le VANET e le MANET condividono molte somiglianze, come la topologia dinamica, la trasmissione dati multi-hop, l'architettura distribuita e la trasmissione omnidirezionale. In entrambe le reti, i nodi mobili possono instradare o rilanciare dati verso la destinazione autonomamente. Tuttavia, ci sono alcune differenze significative tra VANET e MANET. Poiché i veicoli si muovono lungo la strada, la mobilità dei nodi in una VANET è prevedibile, a differenza di una MANET. Inoltre, non ci sono limitazioni in termini di capacità di archiviazione, potenza di elaborazione e durata della batteria dei nodi in una VANET. A causa del rapido movimento dei nodi, la topologia della rete wireless formata è altamente dinamica. Inoltre, la densità della rete in una VANET varia significativamente nel tempo e nello spazio \cite{anwer2014survey}.

Tipicamente, una VANET è composta da tre componenti principali: 
\begin{itemize}
    \item \textbf{On Board Unit (OBU)}: dispositivo embedded inserito all'interno di ogni veicolo comunicante con gli altri mediante un'interfaccia Wireless.
    \item \textbf{Road Side Unit (RSU)}: dispositivo fisso in genere posizionato ai lati della strada; funge da intermediario tra le \textit{On Board Unit} dei veicoli e le infrastrutture stradali o rete Internet.
    \item \textbf{GPS}: sistema di geolocalizzazione.
\end{itemize}

Tutti questi componenti comunicano utilizzando standard/protocolli di comunicazione wireless che determinano vari aspetti della comunicazione, come raggio e velocità di trasmissione dei dati, latenza e sicurezza. La consegna dei dati è considerata una delle sfide principali a causa dei rapidi cambiamenti di topologia, delle frequenti interruzioni del segnale e delle opportunità di contatto nelle VANET. Una VANET può utilizzare diverse tecnologie di rete, come WAVE e IEEE 802.11p, sui quali si baserà il lavoro discusso da questo elaborato, oppure altre tipologie non per forza sviluppate ad-hoc per questo contesto, come WiMAX, Bluetooth e reti cellulari.

\section{Obiettivi}
L'obiettivo principale di questo lavoro sarà lo studio e la realizzazione di un algoritmo DCC (\textit{Decentralized Congestion Control}) in un ambiente Linux per dispositivi che supportano lo standard IEEE 802.11p. Esso verrà svolto mediante la configurazione di un test bed apposito, composto da quattro dispositivi Rock, che verrà descritto successivamente.

Uno degli aspetti critici delle VANET è la gestione della congestione, che può compromettere le prestazioni della rete e, di conseguenza, l'efficacia delle applicazioni di sicurezza e infotainment. In questo contesto, il \textit{Decentralized Congestion Control (DCC)} emerge come una soluzione fondamentale. Il DCC consente ai veicoli di gestire la congestione in modo autonomo, senza la necessità di un controllo centralizzato, regolando dinamicamente il flusso di dati, ottimizzando l'uso delle risorse di rete e garantendo una trasmissione più fluida delle informazioni.

Ci si propone, quindi, di analizzare in dettaglio i protocolli IEEE 802.11p e DCC, esaminando le loro caratteristiche tecniche e il loro funzionamento nel contesto delle VANET, con particolare riferimento agli standard ETSI (European Telecommunications Standards Institute). Attraverso un'analisi delle performance su piattaforme Linux, si valuteranno l'efficacia dei protocolli in differenti scenari, considerando vari parametri di prestazione.

Un aspetto fondamentale di questa ricerca sarà, inoltre, l'integrazione delle metriche di \textit{Quality of Service (QoS)} utilizzando strumenti come iPerf. Questa integrazione permetterà di misurare e ottimizzare le prestazioni della rete, fornendo dati preziosi su throughput e variabilità di esso, che sono essenziali al fine di garantire un'esperienza utente ottimale soprattutto nelle applicazioni di sicurezza ma anche in quelle di infotainment.