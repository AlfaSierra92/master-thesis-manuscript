% https://texblog.org/2013/02/13/latex-documentclass-options-illustrated/
\documentclass[a4paper,12pt,draft,onecolumn,twoside,openright,notitlepage,openany]{book}
% Con openany rimuove le pagine bianche messe apposta per iniziare un nuovo capitolo su pagina bianca (io stampo solo fronte)

\usepackage[utf8]{inputenc}

\usepackage[italian]{babel}
\usepackage{setspace} % per interlinea
\onehalfspacing % interlinea 1.5

\usepackage{lipsum}

% Margini, presi da 
\usepackage[left=3.5cm,right=2.5cm,top=3cm,bottom=3cm,asymmetric]{geometry}
% Uso asymmetric per avere il bordo della rilegatura leggermente più grande

% Times New Roman
\usepackage{times}

% Abilita il supporto alle immagini
\usepackage{graphicx}
%Path relative to the main .tex file 
\graphicspath{ {./images/} }

% Mette i counter della pagina con sezione e sottolineatura
\usepackage{fancyhdr}
\setlength{\headheight}{15pt}
\pagestyle{fancy}

% Li toglie nelle pagine vuote
\usepackage{emptypage}

\begin{document}

\thispagestyle{plain}

\begin{titlepage}
    \begin{center}

		Università degli Studi di Modena e Reggio Emilia\\
		\large
		Dipartimento di Ingegneria "Enzo Ferrari"

        \vspace{2.5cm}
        
		\Large
		Corso di laurea Magistrale in Ingegneria Informatica\\
		
		\large
		Percorso "Cloud and Cyber Security"		
		
		\vspace{3cm}
        
        \Huge
        \textbf{On the performance of Decentralized Congestion Control in a real IEEE 802.11p testbed}

        \vspace{2cm}

        \Large
        \vfill % Aggiungi vfill qui per spingere tutto verso l'alto
        
        % Riga con relatore e studente
        \noindent
        \begin{minipage}[t]{0.5\textwidth}
            \raggedright
            Relatore:\\
            Prof. Carlo Augusto Grazia
        \end{minipage}%
        \begin{minipage}[t]{0.5\textwidth}
            \raggedleft
            Studente:\\
            Antonio Solida
        \end{minipage}

        %\vfill % Lascia uno spazio aggiuntivo per il margine inferiore
        \vspace{2cm}
		
		\small
		Anno Accademico 2023/2024
            
    \end{center}
\end{titlepage}


\thispagestyle{plain}

\begin{center}

\mbox{}
\vfill

\vspace*{\fill}
\begingroup
\centering
\begin{flushright}
\textit{one}

\end{flushright}
\endgroup
\vspace*{\fill}

\newpage

\end{center}


\thispagestyle{plain}
\section*{Abstract}
Negli ultimi due decenni, la sicurezza stradale ha beneficiato di significativi progressi tecnologici, in particolare attraverso lo sviluppo di Intelligent Transport Systems (ITS) e Vehicular Ad-hoc Networks (VANET). Questo lavoro esplora l'importanza di tali sistemi nel migliorare la comunicazione tra veicoli (V2V) e tra veicoli e infrastrutture (V2I), evidenziando, anche, il ruolo cruciale della standardizzazione dei Cooperative Awareness Messages (CAMs). L'implementazione di tecnologie come la Dedicated Short-Range Communication (DSRC) ha facilitato lo scambio di informazioni essenziali per applicazioni di sicurezza, quali quelle deputate alle Collision Avoidance (CA), contribuendo a ridurre gli incidenti stradali.

Il focus principale di questo studio è l'analisi delle politiche di Quality of Service (QoS) in ambienti Linux per dispositivi che supportano lo standard IEEE 802.11p, un aspetto fondamentale per il Decentralized Congestion Control (DCC), in quanto la gestione della congestione si rivela cruciale per garantire l'efficacia delle applicazioni di sicurezza e infotainment nelle VANET. Attraverso un testbed composto da dispositivi OKDO Rockchip 3 Model A, si esamineranno le performance dei protocolli IEEE 802.11p e delle classi di QoS, considerando vari parametri di prestazione. Utilizzando strumenti come iPerf, si valuteranno throughput e variabilità delle prestazioni della rete, fornendo dati essenziali per ottimizzare l'esperienza utente in scenari critici. Questo lavoro intende contribuire alla ricerca nel campo della mobilità intelligente, promuovendo soluzioni innovative per una guida più sicura ed efficiente.


\tableofcontents

\chapter{Introduzione}

\section{Hello World}
Ciao \cite{montanaro2022survey}
\lipsum[1-10]


\input{chapters/02}

\chapter{Qualcosa}

\section{Hello World}
Ciao \cite{shen2013distributed}
\lipsum[1-30]

%\chapter*{Bibliografia1}
\bibliographystyle{ieeetr} % o un altro stile bibliografico di tua scelta
\bibliography{bibliography} % il nome del tuo file .bib

\bibliographystyle{ieeetr} % o un altro stile bibliografico di tua scelta
\bibliography{bibliography}

\chapter*{Ringraziamenti} % * per non numerare il capitolo
Vorrei chiudere questo capitolo della mia vita con una citazione, rischiando di cadere nell'ovvio, ma sicuro del suo valore. Albert Einstein disse: "\textit{Tutti sono dei geni, ma se giudichi un pesce dalla sua capacità di arrampicarsi sugli alberi, passerà la vita a credersi stupido}". Da oggi, mi auguro di smettere di arrampicarmi sugli alberi e, invece, di tuffarmi in mare. Chissà, potrei scoprire di essere un ottimo nuotatore.

Ora, arrivati ai titoli di coda di questa opera "biblica", è arrivato il momento di ringrazire chi c'è stato e chi mi è stato vicino anche da lontano, sicuro che tutti hanno fatto una parte, anche se piccola, nel proseguimento del mio percorso universitario ed anche di vita.

Il primo ringraziamento sento di darlo a me stesso, il quale una primavera di due anni fa, guardandosi indietro, decise di intraprendere un viaggio non del tutto inaspettato per raggiungere nuove vette. Ne è valsa la pena? Mi basta vedere la vista che c'è da qui.

Mi sento in dovere di ringraziare  il \textbf{Prof. Carlo Augusto Grazia}, mio docente ed anche relatore di questa tesi, per la sua professionalità, per la disponibilità che ha concesso me e il supporto datomi in questo lavoro conclusivo del mio percorso universitario magistrale. 

Non posso non spendere due parole per ringraziare mia madre e il mio amico Anthony: anche se non siete qui sono sicuro che sarete fieri di me.

Un ringraziamento va a mio padre, per la fiducia e l'accettazione della mia scelta nonostante la lunga distanza.

Ringrazio i miei cugini Yaya e Pasquale, senza dimenticare la neoarrivata Adele, per i "primi giorni" qui, per i loro consigli e il loro ininterrotto supporto. Ringrazio mia zia Maria e mio zio Gerardo per il loro aiuto, nonostante la distanza. Grazie anche ad Annalisa, Francesca, Alessio, Giulia, Anita e Roberto, qui c'è anche del vostro.

Iniziando i ringraziamenti agli amici, il primo non può che andare a Beatrice: per le lunghe chiacchierate, per la sua amicizia e la fiducia che ha riposto in me chiedendomi consiglio, e per avermi spronato nei momenti di bisogno, specialmente alla vigilia degli esami. Un grazie sincero va anche a voi, cari colleghi, per la vostra amicizia (sono o non sono l'unico e inimitabile Totò?): il compare Davide, compagno di tanti progetti, Diego, Luca “\textit{Palermo}”, Luca “\textit{Abruzzo}”, Marco e Francesco. Un pensiero speciale va anche agli amici di Modena, vecchi e nuovi: Michela, Felicia, Daniel (sì, la tesi l'ho finalmente finita!), Angela, e ovviamente ai miei coinquilini con cui ho condiviso questi anni di vita. Grazie anche agli amici di giù, con cui ho trascorso le poche ferie in Salento. E un ringraziamento speciale va a chi mi ha spronato a “\textit{lasciare la panchina e scendere in campo}” facendomi capire che, invece di continuare ad arrampicarmi, avrei dovuto imparare a nuotare. Senza di voi, alcune scalate sarebbero state davvero troppo difficili da affrontare.

\noindent \textit{Per aspera ad astra!}

\begin{flushright}
    \textbf{\textit{Antonio}}
\end{flushright}

\end{document}
