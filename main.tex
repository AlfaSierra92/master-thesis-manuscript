% https://texblog.org/2013/02/13/latex-documentclass-options-illustrated/
\documentclass[a4paper,12pt,draft,onecolumn,twoside,openright,notitlepage,openany]{book}
% Con openany rimuove le pagine bianche messe apposta per iniziare un nuovo capitolo su pagina bianca (io stampo solo fronte)

\usepackage[utf8]{inputenc}

\usepackage[italian]{babel}
\usepackage{setspace} % per interlinea
\onehalfspacing % interlinea 1.5

\usepackage{lipsum}

% Margini, presi da 
\usepackage[left=3.5cm,right=2.5cm,top=3cm,bottom=3cm,asymmetric]{geometry}
% Uso asymmetric per avere il bordo della rilegatura leggermente più grande

% Times New Roman
\usepackage{times}

% Abilita il supporto alle immagini
\usepackage{graphicx}
%Path relative to the main .tex file 
\graphicspath{ {./images/} }

% Mette i counter della pagina con sezione e sottolineatura
\usepackage{fancyhdr}
\setlength{\headheight}{15pt}
\pagestyle{fancy}

% Li toglie nelle pagine vuote
\usepackage{emptypage}

\begin{document}

\thispagestyle{plain}

\begin{titlepage}
    \begin{center}

		Università degli Studi di Modena e Reggio Emilia\\
		\large
		Dipartimento di Ingegneria "Enzo Ferrari"

        \vspace{2.5cm}
        
		\Large
		Corso di laurea Magistrale in Ingegneria Informatica\\
		
		\large
		Percorso "Cloud and Cyber Security"		
		
		\vspace{3cm}
        
        \Huge
        \textbf{On the performance of Decentralized Congestion Control in a real IEEE 802.11p testbed}

        \vspace{2cm}

        \Large
        \vfill % Aggiungi vfill qui per spingere tutto verso l'alto
        
        % Riga con relatore e studente
        \noindent
        \begin{minipage}[t]{0.5\textwidth}
            \raggedright
            Relatore:\\
            Prof. Carlo Augusto Grazia
        \end{minipage}%
        \begin{minipage}[t]{0.5\textwidth}
            \raggedleft
            Studente:\\
            Antonio Solida
        \end{minipage}

        %\vfill % Lascia uno spazio aggiuntivo per il margine inferiore
        \vspace{2cm}
		
		\small
		Anno Accademico 2023/2024
            
    \end{center}
\end{titlepage}


\thispagestyle{empty}

\begin{center}

\mbox{}


\begin{flushright}
\textit{"To infinity and beyond!"}
\end{flushright}

\newpage

\end{center}


\thispagestyle{plain}
\section*{Abstract}
This is a thesis template for UniPOG.

\newpage

\tableofcontents

\chapter{Introduction}

\section{Hello World}
Ciao

\chapter{Cenni teorici}

\section{Hello World}
Ciao \cite{shen2013distributed}
\lipsum[1-30]

\chapter{Methodology}

\section{Hello World}
Ciao

%\chapter*{Appendice}
appendix
\newpage
\bibliographystyle{ieeetr} % o un altro stile bibliografico di tua scelta
\bibliography{bibliography}

\thispagestyle{plain}

\section*{Ringraziamenti}
Ringrazio mamma, papà ed Emiliano ed Antonio e Solida e Luca e Luca e Feli e Fede etc etc

\end{document}
