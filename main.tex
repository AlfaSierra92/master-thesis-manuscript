% https://texblog.org/2013/02/13/latex-documentclass-options-illustrated/
\documentclass[a4paper,12pt,onecolumn,twoside,openright,notitlepage,openany]{book}
% RIMUOVI DRAFT PER LE IMMAGINI
% Con openany rimuove le pagine bianche messe apposta per iniziare un nuovo capitolo su pagina bianca (io stampo solo fronte)

\usepackage[utf8]{inputenc}

\usepackage[italian]{babel}
\usepackage{setspace} % per interlinea
\onehalfspacing % interlinea 1.5

\usepackage{lipsum}

% Margini, presi da 
\usepackage[left=3.5cm,right=2.5cm,top=3cm,bottom=3cm,asymmetric]{geometry}
% Uso asymmetric per avere il bordo della rilegatura leggermente più grande

% Times New Roman
\usepackage{times}

% Abilita il supporto alle immagini
\usepackage{graphicx}
%Path relative to the main .tex file 
\graphicspath{ {./images/} }

% Mette i counter della pagina con sezione e sottolineatura
\usepackage{fancyhdr}
\setlength{\headheight}{15pt}
\pagestyle{fancy}
\setlength{\parskip}{0pt} % Spaziatura tra paragrafi
\usepackage{titlesec}

% Imposta la spaziatura per le sottosezioni
\titlespacing*{\subsection}{0pt}{15pt}{7pt} % {indent}{spazio prima}{spazio dopo}

\titlespacing*{\section}{0pt}{15pt}{7pt} % {indent}{spazio prima}{spazio dopo}

\usepackage{url} % per citazioni web
\usepackage{color}
\usepackage[final]{listings}
\lstset{ %
aboveskip=10pt,  % Spazio sopra l'ambiente lstlisting
belowskip=10pt,  % Spazio sotto l'ambiente lstlisting
language=C,                % choose the language of the code
basicstyle=\footnotesize,       % the size of the fonts that are used for the code
numbers=none,                   % where to put the line-numbers
numberstyle=\footnotesize,      % the size of the fonts that are used for the line-numbers
stepnumber=1,                   % the step between two line-numbers. If it is 1 each line will be numbered
numbersep=5pt,                  % how far the line-numbers are from the code
backgroundcolor=\color{white},  % choose the background color. You must add \usepackage{color}
showspaces=false,               % show spaces adding particular underscores
showstringspaces=false,         % underline spaces within strings
showtabs=false,                 % show tabs within strings adding particular underscores
frame=single,           % adds a frame around the code
tabsize=2,          % sets default tabsize to 2 spaces
captionpos=b,           % sets the caption-position to bottom
breaklines=true,        % sets automatic line breaking
breakatwhitespace=false,    % sets if automatic breaks should only happen at whitespace
escapeinside={\%*}{*)}          % if you want to add a comment within your code
}


% Li toglie nelle pagine vuote
\usepackage{emptypage}

\usepackage{hyperref} % Per i link nell'indice
\hypersetup{
    colorlinks=true,
    linkcolor=black,
    filecolor=magenta,      
    urlcolor=cyan,
    pdftitle={DCC},
    pdfpagemode=FullScreen,
    }

\begin{document}

\thispagestyle{plain}

\begin{titlepage}
    \begin{center}

		Università degli Studi di Modena e Reggio Emilia\\
		\large
		Dipartimento di Ingegneria "Enzo Ferrari"

        \vspace{2.5cm}
        
		\Large
		Corso di laurea Magistrale in Ingegneria Informatica\\
		
		\large
		Percorso "Cloud and Cyber Security"		
		
		\vspace{3cm}
        
        \Huge
        \textbf{On the performance of Decentralized Congestion Control in a real IEEE 802.11p testbed}

        \vspace{2cm}

        \Large
        \vfill % Aggiungi vfill qui per spingere tutto verso l'alto
        
        % Riga con relatore e studente
        \noindent
        \begin{minipage}[t]{0.5\textwidth}
            \raggedright
            Relatore:\\
            Prof. Carlo Augusto Grazia
        \end{minipage}%
        \begin{minipage}[t]{0.5\textwidth}
            \raggedleft
            Studente:\\
            Antonio Solida
        \end{minipage}

        %\vfill % Lascia uno spazio aggiuntivo per il margine inferiore
        \vspace{2cm}
		
		\small
		Anno Accademico 2023/2024
            
    \end{center}
\end{titlepage}


\thispagestyle{plain}

\begin{center}

\mbox{}
\vfill

\vspace*{\fill}
\begingroup
\centering
\begin{flushright}
\textit{one}

\end{flushright}
\endgroup
\vspace*{\fill}

\newpage

\end{center}


\thispagestyle{plain}
\section*{Abstract}
Negli ultimi due decenni, la sicurezza stradale ha beneficiato di significativi progressi tecnologici, in particolare attraverso lo sviluppo di Intelligent Transport Systems (ITS) e Vehicular Ad-hoc Networks (VANET). Questo lavoro esplora l'importanza di tali sistemi nel migliorare la comunicazione tra veicoli (V2V) e tra veicoli e infrastrutture (V2I), evidenziando, anche, il ruolo cruciale della standardizzazione dei Cooperative Awareness Messages (CAMs). L'implementazione di tecnologie come la Dedicated Short-Range Communication (DSRC) ha facilitato lo scambio di informazioni essenziali per applicazioni di sicurezza, quali quelle deputate alle Collision Avoidance (CA), contribuendo a ridurre gli incidenti stradali.

Il focus principale di questo studio è l'analisi delle politiche di Quality of Service (QoS) in ambienti Linux per dispositivi che supportano lo standard IEEE 802.11p, un aspetto fondamentale per il Decentralized Congestion Control (DCC), in quanto la gestione della congestione si rivela cruciale per garantire l'efficacia delle applicazioni di sicurezza e infotainment nelle VANET. Attraverso un testbed composto da dispositivi OKDO Rockchip 3 Model A, si esamineranno le performance dei protocolli IEEE 802.11p e delle classi di QoS, considerando vari parametri di prestazione. Utilizzando strumenti come iPerf, si valuteranno throughput e variabilità delle prestazioni della rete, fornendo dati essenziali per ottimizzare l'esperienza utente in scenari critici. Questo lavoro intende contribuire alla ricerca nel campo della mobilità intelligente, promuovendo soluzioni innovative per una guida più sicura ed efficiente.


\tableofcontents

\listoffigures

\listoftables

\chapter{Introduzione}

\section{Intelligent Transport System}
Fin dai primi anni del XX secolo, con la nascita dell'industria automobilistica, i produttori di automobili hanno cercato di migliorare i sistemi di sicurezza installati all'interno dei veicoli, ovviamente tutto in funzione di quanto la tecnologia potesse offrire al momento. Si è partiti, ad esempio, dall'introduzione dei freni idraulici su tutte le quattro ruote, per poi passare alle cinture di sicurezza, alla disponibilità degli airbag e all'introduzione dei primi sistemi quali ABS e simili. Ogni nuova tecnologia ha richiesto anni di ricerca e test prima di essere rilasciata e diventare uno standard per i nuovi modelli.

Negli ultimi ventennio, con l'obiettivo di aumentare la sicurezza dei veicoli, sono stati compiuti notevoli sforzi nel campo degli \textit{Intelligent Transport System (ITS)}, anche alla luce di quanto definito dalla direttiva 2010/40/EU dell'Unione Europea \cite{2010-40}. Questo paradigma comprende una serie di applicazioni e servizi utili non solo per la sicurezza del conducente e dei passeggeri, ma anche per la loro esperienza di guida, per l'efficienza del traffico e diverse altre esigenze di trasporto. Le applicazioni legate alla sicurezza e all'efficienza sono interconnesse e si avvantaggiano reciprocamente, motivo per cui l'industria automobilistica, enti governativi e numerosi ricercatori accademici collaborano per standardizzare tutti gli aspetti dei sistemi ITS.

I progressi verso la realizzazione dei sistemi ITS sono stati alimentati da importanti sviluppi nelle \textit{Vehicular Ad-hoc NETworks (VANET)}. Questa tecnologia è stata fondamentale per il successo dei sistemi ITS, consentendo uno scambio rapido e diretto di informazioni necessarie per la maggior parte delle applicazioni ITS. L'introduzione delle VANET, attraverso l'uso della \textit{Dedicated Short-Range Communication (DSRC)}, ha reso possibile lo scambio di messaggi tra Veicolo e Veicolo (V2V) e tra Veicolo e Infrastruttura (V2I).

Uno dei punti cardine, nell'implementazione delle comunicazioni V2V e V2I, è stato quello della standardizzazione, da parte di ETSI, dei cosiddetti \textit{Cooperative Awareness Messages (CAM)} che trova la sua applicazione principale nelle applicazioni deputate alla \textit{Collision Avoidance}, ovvero tutti gli strumenti applicativi atti a prevenire tutta una serie di situazioni di pericolo tra i vari veicoli e tutto ciò che può essere presente sulla sede stradale.

Le CA beneficiano dell'aumento del numero di sensori e della potenza di calcolo presenti nei veicoli moderni, che hanno già portato a funzionalità innovative come la Frenata Automatica d'Emergenza. Tuttavia, queste capacità sono limitate a una visione locale del veicolo; l'idea è di ampliare questa visione condividendo informazioni tramite le VANET con altri veicoli e unità stradali, con l'obiettivo di abilitare applicazioni più complesse ed efficaci. 

Questo implicitamente porta a creare una rete composta da nodi, ovvero veicoli e infrastrutture a bordo strada, che porta a dover affrontare svariati problemi sia implementativi che di performance, i quali richiedono particolari accorgimenti differenti da quelli presi nelle reti tradizionali, in quanto le condizioni di operatività sono completamente diverse.

\section{VANET}
Una \textit{VANET (Vehicular Ad-hoc Network)} è una classe distintiva di \textit{MANET (Mobile Ad-hoc Network)} in cui i veicoli in movimento fungono da nodi o router per scambiare messaggi tra di loro, o come \textit{Access Point (AP)}. Solitamente, una VANET può connettere veicoli entro un raggio di 100-900 metri utilizzando lo standard 802.11p. Il suo obiettivo è supportare sia la comunicazione Veicolo a Veicolo (V2V) che Veicolo a Infrastruttura (V2I) in una rete senza infrastruttura. Sono state avviate numerose iniziative di ricerca, come COOPERS, CVIS, SAFESPOT, PReVENT, Wireless Access in Vehicular Environments (WAVE) e Advanced Safety Vehicle Program (ASV) in Europa, negli Stati Uniti e in Giappone, per rendere gli ITS una realtà. 

Le VANET vengono utilizzate per supportare applicazioni critiche per la sicurezza e applicazioni di intrattenimento non legate alla sicurezza. Le applicazioni di sicurezza, come l'evitamento delle collisioni, la rilevazione pre-collisione o il cambio di corsia, mirano a ridurre gli incidenti stradali attraverso il monitoraggio e la gestione del traffico. Le applicazioni non di sicurezza consentono ai passeggeri di accedere a vari servizi come internet, comunicazione interattiva, giochi online, servizi di pagamento e aggiornamenti informativi mentre i veicoli sono in movimento. La principale differenza tra le applicazioni di sicurezza e quelle non di sicurezza è che le prime sono in grado di inviare e elaborare messaggi in tempo reale. Sia i conducenti che i passeggeri possono accedere a entrambi i tipi di servizi dall'infrastruttura vicina in modo fluido utilizzando tecnologie di accesso wireless.

Le VANET e le MANET condividono molte somiglianze, come la topologia dinamica, la trasmissione dati multi-hop, l'architettura distribuita e la trasmissione omnidirezionale. In entrambe le reti, i nodi mobili possono instradare o rilanciare dati verso la destinazione autonomamente. Tuttavia, ci sono alcune differenze significative tra VANET e MANET. Poiché i veicoli si muovono lungo la strada, la mobilità dei nodi in una VANET è prevedibile, a differenza di una MANET. Inoltre, non ci sono limitazioni in termini di capacità di archiviazione, potenza di elaborazione e durata della batteria dei nodi in una VANET. A causa del rapido movimento dei nodi, la topologia della rete wireless formata è altamente dinamica. Inoltre, la densità della rete in una VANET varia significativamente nel tempo e nello spazio \cite{anwer2014survey}.

Tipicamente, una VANET è composta da tre componenti principali: 
\begin{itemize}
    \item \textbf{On Board Unit (OBU)}: dispositivo embedded inserito all'interno di ogni veicolo comunicante con gli altri mediante un'interfaccia Wireless.
    \item \textbf{Road Side Unit (RSU)}: dispositivo fisso in genere posizionato ai lati della strada; funge da intermediario tra le \textit{On Board Unit} dei veicoli e le infrastrutture stradali o rete Internet.
    \item \textbf{GPS}: sistema di geolocalizzazione.
\end{itemize}

Tutti questi componenti comunicano utilizzando standard/protocolli di comunicazione wireless che determinano vari aspetti della comunicazione, come raggio e velocità di trasmissione dei dati, latenza e sicurezza. La consegna dei dati è considerata una delle sfide principali a causa dei rapidi cambiamenti di topologia, delle frequenti interruzioni del segnale e delle opportunità di contatto nelle VANET. Una VANET può utilizzare diverse tecnologie di rete, come WAVE e IEEE 802.11p, sui quali si baserà il lavoro discusso da questo elaborato, oppure altre tipologie non per forza sviluppate ad-hoc per questo contesto, come WiMAX, Bluetooth e reti cellulari.

\section{Obiettivi}
L'obiettivo principale di questo lavoro sarà lo studio e la realizzazione di un algoritmo DCC (\textit{Decentralized Congestion Control}) in un ambiente Linux per dispositivi che supportano lo standard IEEE 802.11p. Esso verrà svolto mediante la configurazione di un test bed apposito, composto da quattro dispositivi Rock, che verrà descritto successivamente.

Uno degli aspetti critici delle VANET è la gestione della congestione, che può compromettere le prestazioni della rete e, di conseguenza, l'efficacia delle applicazioni di sicurezza e infotainment. In questo contesto, il \textit{Decentralized Congestion Control (DCC)} emerge come una soluzione fondamentale. Il DCC consente ai veicoli di gestire la congestione in modo autonomo, senza la necessità di un controllo centralizzato, regolando dinamicamente il flusso di dati, ottimizzando l'uso delle risorse di rete e garantendo una trasmissione più fluida delle informazioni.

Ci si propone, quindi, di analizzare in dettaglio i protocolli IEEE 802.11p e DCC, esaminando le loro caratteristiche tecniche e il loro funzionamento nel contesto delle VANET, con particolare riferimento agli standard ETSI (European Telecommunications Standards Institute). Attraverso un'analisi delle performance su piattaforme Linux, si valuteranno l'efficacia dei protocolli in differenti scenari, considerando vari parametri di prestazione.

Un aspetto fondamentale di questa ricerca sarà, inoltre, l'integrazione delle metriche di \textit{Quality of Service (QoS)} utilizzando strumenti come iPerf. Questa integrazione permetterà di misurare e ottimizzare le prestazioni della rete, fornendo dati preziosi su throughput e variabilità di esso, che sono essenziali al fine di garantire un'esperienza utente ottimale soprattutto nelle applicazioni di sicurezza ma anche in quelle di infotainment.

\chapter{Cenni teorici}

In questo capitolo, verrà fornita un'introduzione approfondita agli standard e alle tecnologie che sono stati considerati nell'ambito della nostra attività. La comprensione di tali standard è fondamentale per contestualizzare le scelte progettuali e le strategie implementate. Ci concentreremo su diversi protocolli e specifiche tecniche che hanno influenzato lo sviluppo delle soluzioni adottate, esplorando le loro caratteristiche distintive, le modalità di funzionamento e le applicazioni pratiche.

In particolare, analizzeremo gli standard relativi alla comunicazione wireless nell'ambito \textit{automotive}, come WAVE (\textit{Wireless Access in Vehicular Environments}), che gioca un ruolo cruciale nella connettività dei veicoli e nella gestione delle reti di trasporto intelligenti. 

Successivamente, esamineremo il meccanismo EDCA (\textit{Enhanced Distributed Channel Access}), che è una parte integrante del livello MAC (\textit{Media Access Control}) nel contesto delle reti wireless. EDCA introduce un metodo di accesso al canale più sofisticato rispetto al tradizionale DCF (\textit{Distributed Coordination Function}), permettendo una gestione più efficiente delle priorità di traffico. Questo è particolarmente importante in scenari in cui coesistono diversi tipi di traffico, come video, voce e dati, ognuno con requisiti di latenza e larghezza di banda diversi. Discuteremo come EDCA assegna diverse code di accesso per garantire che le comunicazioni più critiche ricevano la priorità necessaria, migliorando così l'esperienza complessiva degli utenti.

Attraverso questa analisi di WAVE e EDCA, il capitolo intende fornire una comprensione approfondita delle tecnologie che supportano le comunicazioni nei veicoli connessi, evidenziando come queste innovazioni possano contribuire a una mobilità più sicura ed efficiente.

%\section{ITS - Intelligent Transport System}

%\section{VANET}

\section{IEEE WAVE}
WAVE è progettato specificamente per ambienti di trasporto, consentendo la comunicazione tra veicoli (\textit{V2V: Vehicle-to-Vehicle}) e tra veicoli e infrastrutture stradali (\textit{V2I: Vehicle-to-Infrastructure}). Questo protocollo sfrutta canali wireless dedicati per garantire una bassa latenza e una maggiore affidabilità, elementi essenziali per applicazioni critiche come la prevenzione degli incidenti e la gestione del traffico. 

Per facilitare questo, l'IEEE ha introdotto un emendamento specifico al protocollo 802.11, noto come 802.11p\cite{std2007wireless}. Questo emendamento si occupa sia del livello fisico, chiamato PHY, sia della gestione dell'accesso al canale, che riguarda il livello MAC. Non ci soffermeremo ulteriormente sui livelli superiori, se non attraverso un breve excursus, poiché il protocollo in questione supporta senza difficoltà qualsiasi tipo di livello, sia esso basato su IP o meno\cite{DSRC-Based-vehicular}. Degno di nota è il fatto che è stato previsto un apposito protocollo per l'invio di frame che non richiedono un livello di trasporto come TCP o UDP, denominato \textit{WAVE Short Message Protocol (WSMP)}.

\begin{figure}[h!]
    \centering
    \includegraphics[width=0.7\textwidth]{WAVE-protocol-stack.png}
    \caption{Stack protocollo WAVE}
    \label{fig:wave_stack}
\end{figure}

Di seguito, presentiamo un elenco che fornisce ulteriori dettagli sui protocolli menzionati nella Figura \ref{fig:wave_stack}, partendo dal layer fisico e salendo di volta in volta:

\begin{itemize}
    \item \textit{IEEE 802.11-2016, IEEE Std 802.11p: Wireless Access in Vehicular Environments}.
    \item \textit{IEEE 1609.0: IEEE Guide for Wireless Access in Vehicular Environments (WAVE) Architecture}, con le sue varie componenti\cite{8686445}: 
        \begin{itemize}
            \item \textit{IEEE 1609.4: IEEE Standard for Wireless Access in Vehicular Environments (WAVE) - Multi-Channel Operation}; per il \textit{channel routing} e il \textit{channel coordination}\cite{7435228}.
            \item \textit{IEEE 1609.3: IEEE Standard for Wireless Access in Vehicular Environments (WAVE) - Networking Services}; per i sopracitati \textit{WAVE Short Messages}\cite{9374154}.
            \item \textit{IEEE 1609.2: IEEE Standard for Wireless Access in Vehicular Environments - Security Services for Applications and Management Messages}; per tutti gli aspetti relativi alla sicurezza\cite{10075082}.
        \end{itemize}
\end{itemize}

Risultano, anche, essere presenti altre varie componenti del protocollo \textit{IEEE 1609.0} su cui non ci si soffermerà e si riportano per completezza:

\begin{itemize}
    \item \textit{IEEE 1609.11:  IEEE Standard for Wireless Access in Vehicular Environments (WAVE) - Over-the-Air Electronic Payment Data}; standard per i pagamenti elettronici in applicazione \textit{WAVE based}\cite{5692959}.
    \item \textit{IEEE 1609.12:  IEEE Standard for Wireless Access in Vehicular Environments (WAVE) - Identifier Allocations}; standard per l'allocazione degli identificatori \textit{WAVE}\cite{8877516}.
\end{itemize}

Per concludere, alla luce del fatto che il contesto preso in esame, quello veicolare, richiede uno scambio di informazioni con latenze minime, \textit{WAVE} si basa interamente su una nuova modalità Wireless, simile alla classica \textit{ad hoc}, chiamata \textit{OCB (Outside Context of a Basic Service Set)}. Questo è un approccio che consente la trasmissione di dati al di fuori delle limitazioni tradizionali delle reti Wi-Fi, permettendo una comunicazione più flessibile e diretta tra i dispositivi, senza la necessità di passare attraverso infrastrutture quali i punti di accesso (\textit{Access Point}).

\subsection[Layer fisico]{Layer fisico}
L'802.11p è uno standard della famiglia IEEE 802.11 progettato specificamente per la comunicazione wireless in ambienti vehicolari. È una tecnologia di rete che consente la comunicazione tra veicoli e tra veicoli e infrastrutture, facilitando applicazioni come la sicurezza stradale, la gestione del traffico e i servizi di infotainment.

\subsection[Layer MAC]{Layer MAC}

\section[EDCA]{EDCA}

\subsection[IEEE 1609.4 for multi-channel operations]{IEEE 1609.4 for multi-channel operations}

\chapter{Testbed}

Per l'implementazione del testbed necessario a eseguire una varietà di test, sono state esaminate numerose schede di sviluppo disponibili sul mercato, caratterizzate da prezzi competitivi e disponibilità immediata. A causa dell'aumento esponenziale dei costi delle varie versioni di Raspberry Pi, soprattutto a seguito della pandemia di Covid-19 e della conseguente scarsità di offerta, si è deciso di escludere a priori queste opzioni.

Inizialmente, si è considerata l'Arduino Yun, una scheda del noto brand Arduino. Tuttavia, si sono subito riscontrato dei limiti invalicabili: le sue prestazioni erano insufficienti e che gli strumenti forniti non garantivano la flessibilità desiderata. In particolare, la shell risultava poco performante e non era possibile impostare tempi di sleep inferiori a un secondo. Inoltre, è importante notare che l'Arduino Yun è stato deprecato a favore di altre schede, che, sebbene più performanti, non offrivano lo stesso rapporto qualità-prezzo.

Successivamente, si è passati all'analisi di router Mikrotik, ma anche in questo caso le prestazioni e la flessibilità nelle configurazioni non hanno soddisfatto le aspettative.

Alla fine, la soluzione ottimale è stata trovata nelle schede Rock 3 Model A. Queste schede, pur essendo leggermente più economiche rispetto ai Raspberry Pi, offrivano le prestazioni, la flessibilità e il costo contenuto ricercati. Sono state utilizzate quattro unità, tutte configurate in modo identico con Armbian OS come sistema operativo. Ulteriori dettagli verranno forniti in seguito.

Per l'esecuzione delle varie prove, sono stati utilizzati due software: iPerf 2 e Netcat versione OpenBSD.

I vari script utilizzati, oltre ad essere riportati in appendice, possono essere trovati nell'apposito \textit{repository Github} \cite{DCCo802.11}.

\section{OKDO Rock 3 Model A board}

L'OKDO Rock 3 Model A (Figura \ref{fig:rock}) è una scheda di sviluppo avanzata, progettata per applicazioni che richiedono elevate prestazioni di calcolo, rendendola particolarmente adatta per progetti di Internet of Things (IoT), edge computing e server leggeri. 

\begin{figure}[h!]
    \centering
    \includegraphics[width=0.7\textwidth]{ROCK_3A.png}
    \caption{Rock 3 Model A}
    \label{fig:rock}
\end{figure}

Al centro di questa scheda si trova il processore Rockchip RK3566, un potente quad-core ARM Cortex-A55 che può raggiungere una frequenza di clock fino a 2.0 GHz. Questa architettura a 64 bit consente di gestire una vasta gamma di applicazioni moderne in modo efficiente. La scheda è dotata di 2 GB di RAM LPDDR4, che garantiscono prestazioni elevate e una gestione efficace delle applicazioni multitasking. Per quanto riguarda l'archiviazione, il Rock 3 Model A offre uno slot per schede microSD, permettendo di espandere facilmente la capacità di memoria, oltre a supportare eMMC fino a 64 GB. Per la connettività, è equipaggiata con una porta Gigabit Ethernet, che assicura una connessione di rete ad alta velocità, e diverse porte USB 3.0 e USB 2.0 per il collegamento di dispositivi esterni. 

Sebbene non sia una caratteristica fondamentale per i nostri scopi, è interessante notare che la scheda include anche una GPU Mali-G52, che consente una buona elaborazione grafica, rendendola adatta per applicazioni multimediali e giochi leggeri. Il Rock 3 Model A offre ampie possibilità di espandibilità, grazie ai pin GPIO (General Purpose Input/Output) che consentono di collegare sensori, attuatori e altri dispositivi. Supporta varie interfacce di comunicazione, come I2C, SPI e UART, facilitando l'integrazione con altri componenti hardware. Per quanto riguarda l'alimentazione, la scheda può essere alimentata tramite un connettore DC o tramite USB-C, offrendo flessibilità nelle opzioni di alimentazione.

La compatibilità con diversi sistemi operativi, tra cui varie distribuzioni Linux come Armbian e Android, rende il Rock 3 Model A estremamente versatile per una varietà di progetti; in particolare, nel nostro caso, si è preferito adottare Armbian OS in quanto risulta essere quello più simile ad un classico sistema \textit{Linux-based} ed inoltre, cosa non di poco conto, fornisce una toolchain comprendente una serie di strumenti e componenti necessari per la compilazione e per il debugging.

\subsection[Atheros Wi-Fi card]{Atheros Wi-Fi card}
Il Rock 3 descritto sopra non monta \textit{out-of-the-box} una scheda di rete che fornisce la connettività Wireless (IEEE 802.11), in virtù di ciò si è reso necessario aggiungerne una esterna collegandola mediante l'interfaccia \textit{PCI Express - M.2 Specification} che la board fornisce per permettere l'espansione delle sue funzionalità hardware.

Tra le varie soluzione in commercio, la scelta è ricaduta sull'adattatore Qualcomm Atheros AR9462. Esso supporta gli standard Wi-Fi 802.11a/b/g/n e permette l'utilizzo della modalità OCB (\textit{Outside the Context of BSS}) utilizzata nel nostro caso al fine di permettere la comunicazione senza fili tra i vari dispositivi senza la necessità di stabilire un \textit{Basic Service Set}.

Un altro aspetto fondamentale che ha reso possibile il nostro lavoro è la possibilità di accedere a una vasta gamma di dati e statistiche sul funzionamento dell'interfaccia di rete, abilitando i flag appropriati durante la compilazione del kernel. Queste informazioni, infatti, non sono generalmente disponibili per la maggior parte delle schede di rete commerciali, rendendo la nostra analisi molto più approfondita e dettagliata. E' possibile accedere a tali statistiche mediante il comando \verb|iw wlp1s0 survey dump| oppure andando ad effettuare un dump dei registri interni della scheda con

\begin{lstlisting}
cat /sys/kernel/debug/ieee80211/phy0/ath9k/regdump
\end{lstlisting}

\noindent Non ci soffermeremo su quest'ultima possibilità in quanto esula dal nostro scopo.

Infine, la natura completamente open-source dei suoi driver ha permesso di apportare ad esso le piccole modifiche necessarie al nostro scopo.

\subsection[Connessione con le board]{Connessione con le board}
Le quattro schede utilizzate nel progetto, come descritto prima, sono dotate di due interfacce di rete distinte, ciascuna con un ruolo specifico nel funzionamento complessivo del sistema. La prima interfaccia è una connessione wireless, utilizzata esclusivamente per l'esecuzione dei test.

La seconda interfaccia è una connessione Ethernet, progettata specificamente per consentire un'interazione diretta con le schede, permettendo così l'invio di comandi per la configurazione, l'avvio e la chiusura dei test, tutti gestiti da un PC.

I quattro dispositivi sono stati collegati a uno switch Ethernet a cinque porte, mentre la porta aggiuntiva è stata riservata per la connessione del PC. Nelle figure \ref{fig:etichetta} e \ref{fig:etichetta_photo} è mostrata la topologia di rete con i vari indirizzi IP assegnati.

\begin{figure}[h!]
    \centering
    \includegraphics[width=0.7\textwidth]{topology.png}
    \caption{Topologia rete}
    \label{fig:etichetta}
\end{figure}

\begin{figure}[h!]
    \centering
    \includegraphics[width=0.5\textwidth]{topology_photo.jpg}
    \caption{Testbed}
    \label{fig:etichetta_photo}
\end{figure}

La connessione avviene mediante il protocollo sicuro SSH, ma non ci si soffermerà su di esso in quanto questo esula dagli obiettivi del presente testo.

\subsection[Impostazione interfaccia Wi-Fi]{Impostazione interfaccia Wi-Fi}
L'interfaccia Wi-Fi di ogni singolo dispositivo Rock, al fine di avere una conessione wireless di tipo OCB funzionante, è stata configurata con i seguenti comandi:

\begin{lstlisting}
    ip link set dev wlp1s0 down
    # imposta la tipologia come ad-hoc
    iw wlp1s0 set type ocb
    # accende l'interfaccia
    ip link set dev wlp1s0 up
    
    iw wlp1s0 ocb join 2462 10MHz
    ip addr add 192.168.100.xxx/24 dev wlp1s0
    ip route add default via 192.168.100.xxx dev wlp1s0
\end{lstlisting}

Non si entrerà nel dettaglio di questi comandi, del resto abbastanza ovvi; ci si limita ad esplicitare che l'interfaccia è stata settata in modalità OCB (\textit{Outside the Context of BSS}), è stato scelto il canale 11, con frequenza fondamentale f\ped{0} pari a 2462 MHz e che, naturalmente, viene assegnato un indirizzo IP con la rispettiva rotta.

Per lo svoglimento del lavoro che viene descritto in questo elaborato, si è deciso di utilizzare una frequenza appartenente alla banda ISM (\textit{Industrial, Scientific and Medical}) anziché quelle specificamente designate dallo standard 802.11p. Questa scelta è stata motivata principalmente dalla necessità di semplificare l'implementazione e garantire la compatibilità hardware.

La banda ISM è ampiamente utilizzata e supportata da numerosi dispositivi commerciali, inclusi quelli a disposizione, il che facilita l'accesso e l'implementazione della tecnologia senza richiedere hardware specializzato. Al contrario, le frequenze stabilite dall'802.11p sono progettate specificamente per le comunicazioni veicolari e richiedono attrezzature più avanzate e costose.

Infatti, l'hardware impiegato nel nostro progetto non era completamente compatibile con le frequenze specifiche dell'IEEE 802.11p, il che avrebbe comportato ulteriori complicazioni e ritardi nell'implementazione.

Optando, invece, per la banda ISM, si è stati in grado di garantire una maggiore interoperabilità e una configurazione più rapida, permettendoci di dedicare maggiore tempo sulla configurazione del nostro sistema e sullo svolgimento delle prove necessarie. A livello pratico, questa scelta non ha comportato sostanziali differenze, se non per il fatto che, lavorando con un canale appartenente alla banda 5.9 GHz anzichè alla 2.4 GHz, si sarebbero potute ottenere prestazioni leggermente migliori grazie a disturbi trascurabili e ad una minore congestione della banda (dovuta a dispositivi terzi che trasmettono sul medesimo canale).

\subsection[Impostazione interfaccia Ethernet]{Impostazione interfaccia Ethernet}
Non ci si soffermerà più di tanto sulla configurazione di tale interfaccia, alla quale è assegnato solo ed esclusivamente un indirizzo IP statico al fine di permettere la raggiungibilità dei dispositivi Rock da parte del PC per l'avvio e lo stop dei vari test.

La configurazione è effettuata da linea di comando mediante il comando \textit{nmtui} (Figura \ref{fig:nmtui}); i vari indirizzi IP assegnati possono essere visti dalla Figura \ref{fig:etichetta}.

\begin{figure}[h!]
    \centering
    \includegraphics[width=0.7\textwidth]{nmtui.png}
    \caption{Comando nmtui}
    \label{fig:nmtui}
\end{figure}

Si fa notare che, purtroppo, i dispositivi Rock hanno tutti il medesimo indirizzo MAC in quanto sono cloni; è stato necessario, quindi, assegnare un nuovo MAC ad ognuno di essi prima di effettuare il collegamento allo switch.

\section{Tweaking}
Per rendere possibili alcune configurazioni nei dispositivi Rock e la raccolta dei dati, si è reso necessario mettere mano nel codice sorgente del Kernel Linux al fine di applicare alcune patch.

Nella fattispecie, si è dovuto provvedere ad abilitare la \textit{Debug Mode} messa a disposizione dal driver \textit{Ath9k} della scheda di rete wireless ma disabilitata in maniera predefinita, ad effettuare una piccola modifica all'output delle statistiche di rete accessibili dall'utente e all'abilitazione delle code separate per le quattro \textit{Access Categories}.

Tutto ciò è stato reso possibile grazie a patch e a informazioni fornite dalla comunità open source e alla documentazione del produttore.

Viene fornita, inizialmente, una descrizione separata delle varie modifiche apportate, esplicitiando alla fine del trittico il processo di compilazione necessario. Importante notare che le varie patch sono state scritte esclusivamente per la versione del kernel Linux \textit{rockchip64-6.6}, l'ultima disponibile al momento della configurazione del testbed; non si garantisce che possano essere applicate senza ulteriori modifiche ad altre versioni, sia che esse siano scritte appositamente per architettura ARM che per altre.

\subsection[Debug Mode]{Debug Mode}
Per l'abilitazione della suddetta modalità, si è semplicemente messo mano al file \textit{linux-rockchip64-current.config}, file di configurazione utilizzato dalla \textit{toolchain} di compilazione del Kernel del sistema Armbian, abilitando tre differenti flag \cite{linux_wireless}:

\verb|CONFIG_ATH9K_HTC_DEBUGFS=y|

\verb|CONFIG_ATH9K_HWRNG=y|

\verb|CONFIG_ATH9K_DEBUGFS=y|

\subsection[Queues]{Queues}
La patch \textit{00550-ac.patch} fornita modifica un segmento di codice nel file wme.c del driver mac80211, che gestisce la classificazione dei pacchetti in una rete wireless.
Molto semplicemente, si è provveduto ad eliminare la classificazione effettuata dal driver \textit{mac80211} in quanto il suo comportamento andava in conflitto con le Access Categories utilizzate e dirottava tutti i pacchetti sulla coda \textit{Best Effort (BE)}, indipendentemente dalla categoria scelta.
Il problema è dovuto all'introduzione delle cosiddette "code software intermedie" all'interno del driver "mac80211", per i driver supportati \cite{intermediate_queue} come l'\textit{Ath9k}.

\begin{lstlisting}
    --- a/net/mac80211/wme.c        2024-06-17 11:16:29
    +++ b/net/mac80211/wme.c        2024-06-17 11:17:12
    @@ -176,9 +176,9 @@
     
            /* use the data classifier to determine what 802.1d tag the
             * data frame has */
    -       qos_map = rcu_dereference(sdata->qos_map);
    -       skb->priority = cfg80211_classify8021d(skb, qos_map ?
    -                                              &qos_map->qos_map : NULL);
    +       //qos_map = rcu_dereference(sdata->qos_map);
    +       //skb->priority = cfg80211_classify8021d(skb, qos_map ?
    +                                              //&qos_map->qos_map : NULL);
     
      downgrade:
            return ieee80211_downgrade_queue(sdata, sta, skb);
\end{lstlisting}
Questa caratteristica è stata introdotta per spostare l'implementazione delle code più verso il lato software del sottosistema wireless, consentendo all'hardware di mantenere solo code brevi e permettendo anche una maggiore equità tra le stazioni che comunicano.

La possibilità di utilizzare correttamente le categorie di accesso insieme alle code software del \textit{mac80211} risultava essere ancora in fase di implementazione da parte della comunità al momento della stesura di questo elaborato.

Per ulteriori dettagli riguardo il sottosistema wireless di Linux e le sue componenti, come ad esempio \textit{mac80211}, vedere \autoref{subsystem}.

\subsection[Statistiche occupazione canale]{Statistiche occupazione canale}
La patch fornita, di nome \textit{00551-ac.patch}, modifica un segmento del codice nel file link.c del driver \textit{ath9k}, che è parte della gestione delle statistiche di monitoraggio della rete wireless. Esso in maniera predefinita restituiva, mediante il comando \verb|iw wlp1s0 survey| \verb|dump|, varie statistiche tra cui i tempi in cui, rispettivamente, il canale risulta essere occupato, viene usato per trasmettere e ricevere da parte dell'interfaccia. 

Questi valori venivano mostrati in output in maniera incrementale, ovvero ogni valore campionato era sommato al valore precedente; tale modus operandi complicava la raccolta dei dati nel nostro caso e si è provveduto a modificare il codice in modo che i valori vengano mostrati, singolarmente, volta per volta.
\begin{lstlisting}
    --- a/drivers/net/wireless/ath/ath9k/link.c	2024-06-21 11:46:11
    +++ b/drivers/net/wireless/ath/ath9k/link.c	2024-06-21 11:47:38
    @@ -524,10 +524,10 @@
                 SURVEY_INFO_TIME_BUSY |
                 SURVEY_INFO_TIME_RX |
                 SURVEY_INFO_TIME_TX;
    -		survey->time += cc->cycles / div;
    -		survey->time_busy += cc->rx_busy / div;
    -		survey->time_rx += cc->rx_frame / div;
    -		survey->time_tx += cc->tx_frame / div;
    +		survey->time = cc->cycles / div;
    +		survey->time_busy = cc->rx_busy / div;
    +		survey->time_rx = cc->rx_frame / div;
    +		survey->time_tx = cc->tx_frame / div;
         }
     
         if (cc->cycles < div)
\end{lstlisting}
Giusto per completezza, viene mostrato l'output dell'esecuzione del comando 

\verb|iw| \verb|wlp1s0 survey dump|:
\begin{lstlisting}
root@rock-3a:~# iw wlp1s0 survey dump
    Survey data from wlp1s0
	frequency:			2462 MHz [in use]
	noise:				-93 dBm
	channel active time:		136546 ms
	channel busy time:		1642 ms
	channel receive time:		1496 ms
	channel transmit time:		0 ms
\end{lstlisting}
Per semplicità è riportato qui solo l'output relativo al canale 11, omettendo le informazioni relative agli altri canali.

Per calcolare il carico del canale, ovvero la percentuale di tempo in cui il canale wireless è occupato, sono stati raccolti due valori fondamentali: il \textit{channel active time} e il \textit{channel busy time}. Il primo rappresenta il tempo in cui l'interfaccia è attiva per trasmettere, ricevere o ascoltare il canale, mentre il secondo indica il tempo in cui l'interfaccia è attiva ma in attesa a causa dell'occupazione del canale \cite{han2016adaptive}.

\subsection[Kernel building]{Kernel building}
Per la compilazione dell'intero sistema Armbian o, nello specifico, del proprio Kernel, ci si è basati sulla toolchain messaci a disposizione dalla comunità open source \cite{armbian_tool}. 

La toolchain di Armbian è una componente essenziale per sviluppare e ottimizzare software per dispositivi ARM, specialmente quando si lavora con macchine host x86. Il cuore di questa toolchain è il compilatore GCC, configurato specificamente per il target ARM. Questo compilatore traduce il codice sorgente in codice eseguibile ottimizzato per le architetture ARM, permettendo di sfruttare appieno le capacità hardware dei dispositivi.

Un suo aspetto cruciale è la cross-compilazione. Lavorare con dispositivi ARM spesso significa confrontarsi con limitazioni di risorse hardware che rendono impraticabile la compilazione diretta. La cross-compilazione risolve questo problema permettendo di compilare il software su una macchina più potente (solitamente un PC x86) e poi eseguirlo sul dispositivo ARM. Questo processo non solo accelera notevolmente la compilazione, ma evita anche il rischio di esaurire le risorse durante la build.

Il sistema di build automatizzato di Armbian gioca un ruolo centrale nella gestione di questo processo. Gli script e gli strumenti inclusi orchestrano la compilazione del kernel, del bootloader e dei vari pacchetti software, assicurando che tutto sia perfettamente integrato e ottimizzato. Questo sistema è pensato per gestire le complessità della cross-compilazione, semplificando il processo per gli sviluppatori.

Non ci si dilungherà nelle modalità di download e installazione di tutto il pacchetto, in quanto è tranquillamente reperibile in rete e richiede, per la sua configurazione iniziale, componenti esterne quali Docker. Una volta installato e funzionante, fatte le modifiche ad-hoc al file di configurazione (il citato precedentemente file \textit{linux-rockchip64-current.config}) ed inserite le patch già menzionate nella cartella

\verb|~/build/patch/kernel/archive/rockchip64-6.6 |

\noindent si può procedere alla compilazione mediante il comando 

\verb|./compile.sh kernel ARTIFACT_IGNORE_CACHE='yes' \| 

\verb|BOARD=rock-3a BRANCH=current|

\noindent Il processo restituirà il nuovo kernel con i relativi \textit{header} pacchettizzati all'interno di file \textit{.deb}, che andranno installati sui dispositivi come di consueto.

\subsection{Sottosistema Wireless in Linux}
\label{subsystem}
Questa sottosezione è stata redatta per fornire un contesto adeguato e chiarire alcuni concetti emersi nei paragrafi precedenti riguardanti il sottosistema wireless di Linux e la componente \textit{mac80211} \cite{short_walk}.

Sempre più interfacce wireless adottano l'approccio SoftMAC, che consente un controllo più preciso della gestione dei frame tramite software. Il framework \textit{mac80211} è stato introdotto e integrato nei recenti kernel Linux per fornire funzionalità comuni del livello MAC per la gestione dei frame, come la generazione e il parsing, evitando la dipendenza da driver wireless individuali \cite{wireless-profiling}; in parole povere, la composizione del frame MAC e la parte di accesso al canale viene fatta via software dal driver anzichè via hardware dall'interfaccia.

\begin{figure}[h!]
    \centering
    \includegraphics[width=0.7\textwidth]{wireless-sub.png}
    \caption{Sottosistema wireless in Linux}
    \label{fig:wireless-sub}
\end{figure}

La Figura \ref{fig:wireless-sub} mostra lo stack wireless di un sistema operativo \textit{Linux-based}, in cui il framework è parte integrante del kernel. Esso fornisce configurazione e gestione della radio wireless e della trasmissione dei dati attraverso strumenti dello \textit{user space}, come lo \textit{utility daemon} wireless, \textit{iw}. Insieme a \textit{mac80211}, è stato introdotto anche il componente \textit{cfg80211}, che gestisce le operazioni relative alla configurazione dell'hardware sottostante tramite l'interfaccia \textit{cfg80211\_ops}. La comunicazione tra gli strumenti dello \textit{user space} e \textit{cfg80211} avviene attraverso il socket netlink, \textit{nl80211}. Tra il framework e i dispositivi fisici si trovano driver individuali, come \textit{ath5k} (\textit{ath9k} nei dipositivi del testbed), che implementano funzioni comuni per il controllo dell'hardware, come definito dall'interfaccia \textit{ieee80211\_ops}. Quest'ultima altro non è che una \verb|struct| contenente un buon numero di puntatori a funzioni, collegate a "procedure" di basso livello eseguite dalla scheda di rete.

Il framework funge da middleware tra l'hardware fisico e gli strumenti dello \textit{user space}, fornendo un'astrazione per dispositivi wireless con specifiche diverse. Inoltre, offre funzionalità comuni per i singoli driver, semplificando notevolmente lo sviluppo di nuovi driver per dispositivi SoftMAC.

Oltre alla gestione dei frame, il framework fornisce meccanismi per il controllo della velocità del livello MAC. Pertanto, i driver wireless possono utilizzare i meccanismi di controllo della velocità forniti o implementare soluzioni proprie \cite{wireless-profiling}.


\section{iPerf con Access Categories abilitate}
iPerf è un software di benchmarking di rete utilizzato per misurare le prestazioni della larghezza di banda e della latenza nelle connessioni TCP e UDP. Permette di misurare la larghezza di banda disponibile tra due punti di rete, offrendo la possibilità di testare sia il protocollo TCP, orientato alla connessione, sia il protocollo UDP, non orientato alla connessione. Il software funziona in modalità client-server, dove un'istanza di iPerf agisce da server e l'altra da client, permettendo di eseguire test in entrambe le direzioni.

Le modalità con cui tale software è stato utilizzato verranno discusse più avanti nell'apposito capitolo; qui ci si soffermerà, più che altro, sulle modifiche effettuate al codice sorgente.

La patch, che per chiarezza di lettura non è stata inserita in questo paragrafo ma potrà essere trovata in appendice (\autoref{iperf_ac}), abilita il supporto alle 4 code EDCA del livello MAC. L'opzione -A, specifica per iPerf in modalità client, può ora essere usata per specificare una classe di traffico a cui inviare il flusso in uscita (-A BK o -A BE o -A VI o -A VO). Se non si specifica alcuna classe di traffico, le opzioni rimangono quelle di un pacchetto iPerf 2 standard (cioè si utilizza effettivamente AC\_BE).

Da notare che la patch è stata sviluppata per la versione 2 del software (nello specifico la v2.2.0); esiste anche la versione 3 che, tuttavia, non è stata presa in considerazione in quanto i cambiamenti architetturali effettuati al programma non consentivano, facilmente, l'impletazione delle modifiche necessarie.

\subsection[Parametri utilizzati]{Parametri utilizzati}
Vengono qui elencati e brevemente spiegati alcuni dei parametri utilizzati per le varie esecuzioni di iPerf \cite{iperf}.

\begin{itemize}
    \item \textit{-s}: esegue iPerf in modalità server.
    \item \textit{-u}: usa UDP piuttosto che TCP. Se non inserito, viene utilizzato quest'ultimo in maniera predefinita.
    \item \textit{-c aaa.bbb.ccc.ddd}: eseguire iPerf in modalità client, collegandosi a un server iPerf in esecuzione su un host con l'indirizzo IP passato come parametro.
    \item \textit{-i n}: imposta il tempo di intervallo in secondi tra la larghezza di banda periodica, il jitter e i rapporti di perdita. Se diverso da zero, viene fatto un rapporto ogni intervallo di secondi della larghezza di banda dall'ultimo rapporto. Se zero, non vengono stampati rapporti periodici. Il valore predefinito è zero.
    \item \textit{-t n}: il tempo in secondi per cui trasmettere.
    \item \textit{-b n}: impostare la larghezza di banda di destinazione su n bit/sec.
    \item \textit{-A xx}: classe di priorità da utilizzare; già spiegato precedentemente.
    \item \textit{-y}: passandoci assieme il carattere C si avrà l'output come CSV (\textit{Comma Separated Value}).
    \item \textit{-o filename}: output su file anzichè su \textit{stdout}.
    \item \textit{-w 416K}: imposta le dimensioni del buffer del socket sul valore specificato (416 KB). Per TCP, questo imposta la dimensione della finestra TCP.
\end{itemize}

\section{Netcat OpenBSD}
Netcat, spesso abbreviato in nc, è un potente strumento di rete utilizzato per leggere e scrivere dati attraverso connessioni di rete TCP o UDP. È spesso descritto come "il coltellino svizzero" degli strumenti di rete, grazie alla sua versatilità.

Netcat può essere utilizzato per una varietà di scopi, tra cui il debugging delle connessioni di rete, il trasferimento di file, la creazione di tunnel e la comunicazione tra sistemi. Può funzionare sia in modalità client, per connettersi a un server, sia in modalità server, per ascoltare le connessioni in arrivo su una porta specifica.

Esistono differenti porting e implementazioni; le più importanti sono GNU e OpenBSD; esse sono simili se non per alcune differenze nelle funzionalità offerte e in alcuni parametri di esecuzione disponibili (la versione BSD, per esempio, supporta IPv6, proxy e gli Unix socket che la tradizionale non ha). 

Nel nostro caso, la necessità di dover eseguire il comando netcat col paramentro \verb|-q 0|, ovvero disabilitare l'attesa dopo l'EOF su stdin, ha portato ad utilizzare l'OpenBSD anzichè la tradizionale.


\chapter{Misurazioni e risultati}

\section{Tx\_Cx}
Ciao

\section{Test 2}

\section{Lost packets}

\chapter{Conclusioni}

\section{Hello World}
\lipsum[1-10]

\chapter{Appendice}

\section{Patch iPerf per AC}
\label{iperf_ac}
\begin{lstlisting}
    --- a/include/Settings.hpp
    +++ b/include/Settings.hpp
    @@ -135,6 +135,7 @@ typedef struct thread_Settings {
         // int's
         int mThreads;                   // -P
         int mTOS;                       // -S
    +    int mMACUP;                     // -A
     #if WIN32
         SOCKET mSock;
     #else
    --- a/include/version.h
    +++ b/include/version.h
    @@ -1,4 +1,4 @@
    -#define IPERF_VERSION "2.0.13"
    +#define IPERF_VERSION "2.0.13 OpenWrt-V2X patch"
     #define IPERF_VERSION_DATE "21 Jan 2019"
     #define IPERF_VERSION_MAJORHEX 0x00020000
     #define IPERF_VERSION_MINORHEX 0x000D0003
    --- a/src/Locale.c
    +++ b/src/Locale.c
    @@ -103,6 +103,7 @@ Server specific:\n\
       -s, --server             run in server mode\n\
       -t, --time      #        time in seconds to listen for new connections as well as to receive traffic (default not set)\n\
           --udp-histogram #,#  enable UDP latency histogram(s) with bin width and count, e.g. 1,1000=1(ms),1000(bins)\n\
    +  -A, --accesscategory <AC> Forces a certain EDCA MAC access category to be used (BK, BE, VI, VO)\n\
       -B, --bind <ip>[%<dev>]  bind to multicast address and optional device\n\
       -H, --ssm-host <ip>      set the SSM source, use with -B for (S,G) \n\
       -U, --single_udp         run in single threaded UDP mode\n\
    @@ -125,6 +126,7 @@ Client specific:\n\
     "  -n, --num       #[kmgKMG]    number of bytes to transmit (instead of -t)\n\
       -r, --tradeoff           Do a bidirectional test individually\n\
       -t, --time      #        time in seconds to transmit for (default 10 secs)\n\
    +  -A, --accesscategory <AC> Forces a certain EDCA MAC access category to be used (BK, BE, VI, VO)\n\
       -B, --bind [<ip> | <ip:port>] bind ip (and optional port) from which to source traffic\n\
       -F, --fileinput <name>   input the data to be transmitted from a file\n\
       -I, --stdin              input the data to be transmitted from stdin\n\
    --- a/src/PerfSocket.cpp
    +++ b/src/PerfSocket.cpp
    @@ -155,6 +155,15 @@ void SetSocketOptions( thread_Settings *
         }
     #endif
     
    +   	// set MAC AC (access category) field, if specified only (i.e. if mMACUP != -1)
    +    // AC is set starting from user priorities (UP)
    +	if ( inSettings->mMACUP >= 0 ) {
    +		int  up = inSettings->mMACUP;
    +		Socklen_t len = sizeof(up);
    +		int rc = setsockopt( inSettings->mSock, SOL_SOCKET, SO_PRIORITY, (char*) &up, len );
    +		WARN_errno( rc == SOCKET_ERROR, "setsockopt SO_PRIORITY" );
    +	}
    +
         if ( !isUDP( inSettings ) ) {
             // set the TCP maximum segment size
             setsock_tcp_mss( inSettings->mSock, inSettings->mMSS );
    --- a/src/Settings.cpp
    +++ b/src/Settings.cpp
    @@ -131,6 +131,7 @@ const struct option long_options[] =
     {"realtime",         no_argument, NULL, 'z'},
     
     // more esoteric options
    +{"accesscategory",  required_argument, NULL, 'A'},
     {"bind",       required_argument, NULL, 'B'},
     {"compatibility",    no_argument, NULL, 'C'},
     {"daemon",           no_argument, NULL, 'D'},
    @@ -198,6 +199,7 @@ const struct option env_options[] =
     {"IPERF_REPORTSTYLE",required_argument, NULL, 'y'},
     
     // more esoteric options
    +{"IPERF_MACAC",        required_argument, NULL, 'A'},
     {"IPERF_BIND",       required_argument, NULL, 'B'},
     {"IPERF_COMPAT",           no_argument, NULL, 'C'},
     {"IPERF_DAEMON",           no_argument, NULL, 'D'},
    @@ -218,7 +220,8 @@ const struct option env_options[] =
     
     #define SHORT_OPTIONS()
     
    -const char short_options[] = "1b:c:def:hi:l:mn:o:p:rst:uvw:x:y:zB:CDF:H:IL:M:NP:RS:T:UVWXZ:";
    +// Edited to add the A: (A + 1 argument) short option
    +const char short_options[] = "1b:c:def:hi:l:mn:o:p:rst:uvw:x:y:zA:B:CDF:IL:M:NP:RS:T:UVWZ:";
     
     /* ------------------------------------------------------
      * defaults
    @@ -279,6 +282,7 @@ void Settings_Initialize( thread_Setting
         //main->mThreads      = 0;           // -P,
         //main->mRemoveService = false;      // -R,
         //main->mTOS          = 0;           // -S,  ie. don't set type of service
    +    main->mMACUP        = -1;            // -A (set to an invalid number as default -> with -1 no setsockopt will be called for AC)
         main->mTTL          = -1;            // -T,  link-local TTL
         //main->mDomain     = kMode_IPv4;    // -V,
         //main->mSuggestWin = false;         // -W,  Suggest the window size.
    @@ -692,6 +696,24 @@ void Settings_Interpret( char option, co
                 mExtSettings->mTOS = strtol( optarg, NULL, 0 );
                 break;
     
    +        case 'A': // 802.11p/802.11e MAC layer access categories
    +            // Mapping between UP (0 to 7) and AC (BK to VO)
    +            if(strcmp(optarg,"BK") == 0) {
    +                mExtSettings->mMACUP=1; // UP=1 (2) is AC_BK
    +            } else if(strcmp(optarg,"BE") == 0) {
    +                mExtSettings->mMACUP=0; // UP=0 (3) is AC_BE
    +            } else if(strcmp(optarg,"VI") == 0) {
    +                mExtSettings->mMACUP=4; // UP=4 (5) is AC_VI
    +            } else if(strcmp(optarg,"VO") == 0) {
    +                mExtSettings->mMACUP=6; // UP=6 (7) is AC_VO
    +            } else {
    +                // Leave to default (-1), i.e. no AC is set to socket, and print error
    +                fprintf(stderr, "Invalid AC specified with -A\nValid ones are: BK, BE, VI, VO\nNo AC will be set\n");
    +            }
    +
    +            //mExtSettings->mMACUP = strtol( optarg, NULL, 0 );
    +            break;
    +
             case 'T': // time-to-live for both unicast and multicast
                 mExtSettings->mTTL = atoi( optarg );
                 break;
\end{lstlisting}

\section{Invio pacchetti UDP}
\label{udp_packets}
\begin{lstlisting}
     #!/bin/bash

     # Verifica che siano stati forniti i tre argomenti necessari
     if [ "$#" -ne 2 ]; then
         echo "Usage: $0 <message_size> <sleep_time>"
         exit 1
     fi
     
     message_size=$1
     sleep_time=$2
     ip_broadcast=192.168.100.255
     interface=wlp1s0
     random_port=$(( ( RANDOM % 64512 ) + 1024 ))
     
     # Genera un messaggio di dimensione specificata
     message=$(head -c $message_size </dev/zero | tr '\0' 'A')
     
     # Inizia lo script
     echo "Starting..."
     sleep 1
     
     while true; do
         # Invio dei messaggi
         echo -n "$message" | nc -u -b -w1 -q0 $ip_broadcast $random_port &
         sleep $sleep_time
     done
\end{lstlisting}

\section{Plot media valori Tx\_Cx}
\label{plot_test1}
\begin{lstlisting}
     import pandas as pd
     import matplotlib.pyplot as plt
     import seaborn as sns
     
     # Imposta lo stile di Seaborn
     sns.set(style="whitegrid")
     
     # Lista dei file CSV
     csv_files = ['Data/outputT1_C0_1.csv', 'Data/outputT1_C0_2.csv', 'Data/outputT1_C0_3.csv', 'Data/outputT1_C0_4.csv', 'Data/outputT1_C0_5.csv']
     
     # Inizializza liste per i dati filtrati
     filtered_data_list = []
     
     # Leggi i file CSV, filtra e calcola il throughput
     for file in csv_files:
         data = pd.read_csv(file, header=None)
         data.columns = ['timestamp', 'src_ip', 'src_port', 'dst_ip', 'dst_port', 'flow_id', 'time_interval', 'bytes', 'packets']
         data['timestamp'] = pd.to_datetime(data['timestamp'], format='%Y%m%d%H%M%S')
         data['throughput_mbps'] = (data['bytes'] * 8) / 1_000_000
         start_time = data['timestamp'].min()
         start_time_1 = start_time + pd.Timedelta(seconds=20)
         end_time = start_time + pd.Timedelta(seconds=50)
         filtered_data = data[(data['timestamp'] >= start_time_1) & (data['timestamp'] < end_time)].copy()  # Usa .copy()
         filtered_data_list.append(filtered_data)
     
     # Calcola la media del throughput per ciascun secondo per ogni stream ID
     mean_throughput_1 = []
     mean_throughput_2 = []
     for second in range(30):
         second_data_1 = []
         second_data_2 = []
         for filtered_data in filtered_data_list:
             timestamp_second = start_time + pd.Timedelta(seconds=second + 20)  # Aggiungi i 25 secondi di offset
             stream_id_1 = filtered_data[(filtered_data['flow_id'] == 1) & (filtered_data['timestamp'].dt.floor('s') == timestamp_second)]
             stream_id_2 = filtered_data[(filtered_data['flow_id'] == 2) & (filtered_data['timestamp'].dt.floor('s') == timestamp_second)]
             if not stream_id_1.empty:
                 second_data_1.append(stream_id_1['throughput_mbps'].mean())
             if not stream_id_2.empty:
                 second_data_2.append(stream_id_2['throughput_mbps'].mean())
         mean_throughput_1.append(sum(second_data_1) / len(second_data_1) if second_data_1 else 0)
         mean_throughput_2.append(sum(second_data_2) / len(second_data_2) if second_data_2 else 0)
     
     # Calcola la media e la varianza del throughput per ciascun stream ID
     throughput_avg_1 = sum(mean_throughput_1) / len(mean_throughput_1)
     throughput_var_1 = sum((x - throughput_avg_1) ** 2 for x in mean_throughput_1) / len(mean_throughput_1)
     
     throughput_avg_2 = sum(mean_throughput_2) / len(mean_throughput_2)
     throughput_var_2 = sum((x - throughput_avg_2) ** 2 for x in mean_throughput_2) / len(mean_throughput_2)
     
     # Print overall results
     print(f'Throughput Medio Totale per Stream ID 1: {throughput_avg_1:.2f} Mbps')
     print(f'Varianza Totale per Stream ID 1: {throughput_var_1:.6f} Mbps^2')
     print(f'Throughput Medio Totale per Stream ID 2: {throughput_avg_2:.2f} Mbps')
     print(f'Varianza Totale per Stream ID 2: {throughput_var_2:.6f} Mbps^2')
     
     # Plot del throughput medio per stream ID 1
     plt.figure(figsize=(12, 6))
     plt.subplot(2, 1, 1)
     plt.plot(range(30), mean_throughput_1, linestyle='-', color='blue', markersize=8, label='Stream ID 1')
     plt.title(f'Throughput Medio per Stream ID 1', fontsize=16)
     plt.xlabel('Secondi', fontsize=14)
     plt.ylabel('Throughput Medio (Mbps)', fontsize=14)
     plt.grid(True, linestyle='--', alpha=0.7)
     plt.legend()
     
     # Plot del throughput medio per stream ID 2
     plt.subplot(2, 1, 2)
     plt.plot(range(30), mean_throughput_2, linestyle='-', color='orange', markersize=8, label='Stream ID 2')
     plt.title(f'Throughput Medio per Stream ID 2', fontsize=16)
     plt.xlabel('Secondi', fontsize=14)
     plt.ylabel('Throughput Medio (Mbps)', fontsize=14)
     plt.grid(True, linestyle='--', alpha=0.7)
     plt.legend()
     
     plt.tight_layout()
     plt.show()     
\end{lstlisting}

\section{Plot Ensemble Tx\_Cx}
\label{plot_test_spaghetti}
\begin{lstlisting}
     import pandas as pd
     import matplotlib.pyplot as plt
     import seaborn as sns
     
     # Imposta lo stile di Seaborn
     sns.set(style="whitegrid")
     
     # Lista dei file CSV
     csv_files = ['Data/outputT1_C0_1.csv', 'Data/outputT1_C0_2.csv', 'Data/outputT1_C0_3.csv', 'Data/outputT1_C0_4.csv', 'Data/outputT1_C0_5.csv']
     
     # Colori e stili di linea per i plot
     colors = ['blue', 'orange', 'green', 'red', 'purple']
     line_styles = ['-', '--', '-.', ':', '-']
     
     # Inizializza liste per calcolare media e varianza
     all_stream_id_1 = []
     all_stream_id_2 = []
     
     # Leggi i file CSV, filtra e calcola il throughput
     filtered_data_list = []
     for file in csv_files:
         data = pd.read_csv(file, header=None)
         data.columns = ['timestamp', 'src_ip', 'src_port', 'dst_ip', 'dst_port', 'flow_id', 'time_interval', 'bytes', 'packets']
         data['timestamp'] = pd.to_datetime(data['timestamp'], format='%Y%m%d%H%M%S')
         data['throughput_mbps'] = (data['bytes'] * 8) / 1_000_000
         start_time = data['timestamp'].min()
         start_time_1 = start_time + pd.Timedelta(seconds=20)
         end_time = start_time + pd.Timedelta(seconds=50)
         filtered_data = data[(data['timestamp'] >= start_time_1) & (data['timestamp'] < end_time)].copy()  # Usa .copy()
         filtered_data_list.append(filtered_data)
         all_stream_id_1.append(filtered_data[filtered_data['flow_id'] == 1])
         all_stream_id_2.append(filtered_data[filtered_data['flow_id'] == 2])
     
     # Concatenate all data for stream ID 1 and 2
     all_stream_id_1 = pd.concat(all_stream_id_1)
     all_stream_id_2 = pd.concat(all_stream_id_2)
     
     # Plot del throughput medio per stream ID 1
     plt.figure(figsize=(12, 6))
     plt.subplot(2, 1, 1)
     for i, (filtered_data, color, line_style) in enumerate(zip(filtered_data_list, colors, line_styles)):
         stream_id_1 = filtered_data[filtered_data['flow_id'] == 1]
         throughput_avg_1_grouped = stream_id_1.groupby(stream_id_1['timestamp'].dt.floor('s'))['throughput_mbps'].mean()
         sample_indices_1 = range(len(throughput_avg_1_grouped))
         plt.plot(sample_indices_1, throughput_avg_1_grouped.values, linestyle=line_style, color=color, markersize=8, label=f'Run {i+1}')
     plt.title(f'Throughput Medio per Stream ID 1', fontsize=16)
     plt.xlabel('Secondi', fontsize=14)
     plt.ylabel('Throughput Medio (Mbps)', fontsize=14)
     plt.grid(True, linestyle='--', alpha=0.7)
     plt.legend(loc='center left', bbox_to_anchor=(1, 0.5))  # Legenda fuori dal grafico
     
     # Plot del throughput medio per stream ID 2
     plt.subplot(2, 1, 2)
     for i, (filtered_data, color, line_style) in enumerate(zip(filtered_data_list, colors, line_styles)):
         stream_id_2 = filtered_data[filtered_data['flow_id'] == 2]
         throughput_avg_2_grouped = stream_id_2.groupby(stream_id_2['timestamp'].dt.floor('s'))['throughput_mbps'].mean()
         sample_indices_2 = range(len(throughput_avg_2_grouped))
         plt.plot(sample_indices_2, throughput_avg_2_grouped.values, linestyle=line_style, color=color, markersize=8, label=f'Run {i+1}')
     plt.title(f'Throughput Medio per Stream ID 2', fontsize=16)
     plt.xlabel('Secondi', fontsize=14)
     plt.ylabel('Throughput Medio (Mbps)', fontsize=14)
     plt.grid(True, linestyle='--', alpha=0.7)
     plt.legend(loc='center left', bbox_to_anchor=(1, 0.5))  # Legenda fuori dal grafico
     
     plt.tight_layout()
     plt.show()     
\end{lstlisting}

%\chapter*{Bibliografia1}
\bibliographystyle{ieeetr} % o un altro stile bibliografico di tua scelta
\bibliography{bibliography} % il nome del tuo file .bib

\bibliographystyle{unsrt} % o un altro stile bibliografico di tua scelta
\bibliography{bibliography}

\chapter*{Ringraziamenti} % * per non numerare il capitolo
Vorrei chiudere questo capitolo della mia vita con una citazione, rischiando di cadere nell'ovvio, ma sicuro del suo valore. Albert Einstein disse: "\textit{Tutti sono dei geni, ma se giudichi un pesce dalla sua capacità di arrampicarsi sugli alberi, passerà la vita a credersi stupido}". Da oggi, mi auguro di smettere di arrampicarmi sugli alberi e, invece, di tuffarmi in mare. Chissà, potrei scoprire di essere un ottimo nuotatore.

Ora, arrivati ai titoli di coda di questa opera "biblica", è arrivato il momento di ringrazire chi c'è stato e chi mi è stato vicino anche da lontano, sicuro che tutti hanno fatto una parte, anche se piccola, nel proseguimento del mio percorso universitario ed anche di vita.

Il primo ringraziamento sento di darlo a me stesso, il quale una primavera di due anni fa, guardandosi indietro, decise di intraprendere un viaggio non del tutto inaspettato per raggiungere nuove vette. Ne è valsa la pena? Mi basta vedere la vista che c'è da qui.

Mi sento in dovere di ringraziare  il \textbf{Prof. Carlo Augusto Grazia}, mio docente ed anche relatore di questa tesi, per la sua professionalità, per la disponibilità che ha concesso me e il supporto datomi in questo lavoro conclusivo del mio percorso universitario magistrale. 

Non posso non spendere due parole per ringraziare mia madre e il mio amico Anthony: anche se non siete qui sono sicuro che sarete fieri di me.

Un ringraziamento va a mio padre, per la fiducia e l'accettazione della mia scelta nonostante la lunga distanza.

Ringrazio i miei cugini Yaya e Pasquale, senza dimenticare la neoarrivata Adele, per i "primi giorni" qui, per i loro consigli e il loro ininterrotto supporto. Ringrazio mia zia Maria e mio zio Gerardo per il loro aiuto, nonostante la distanza. Grazie anche ad Annalisa, Francesca, Alessio, Giulia, Anita e Roberto, qui c'è anche del vostro.

Iniziando i ringraziamenti agli amici, il primo non può che andare a Beatrice: per le lunghe chiacchierate, per la sua amicizia e la fiducia che ha riposto in me chiedendomi consiglio, e per avermi spronato nei momenti di bisogno, specialmente alla vigilia degli esami. Un grazie sincero va anche a voi, cari colleghi, per la vostra amicizia (sono o non sono l'unico e inimitabile Totò?): il compare Davide, compagno di tanti progetti, Diego, Luca “\textit{Palermo}”, Luca “\textit{Abruzzo}”, Marco e Francesco. Un pensiero speciale va anche agli amici di Modena, vecchi e nuovi: Michela, Felicia, Daniel (sì, la tesi l'ho finalmente finita!), Angela, e ovviamente ai miei coinquilini con cui ho condiviso questi anni di vita. Grazie anche agli amici di giù, con cui ho trascorso le poche ferie in Salento. E un ringraziamento speciale va a chi mi ha spronato a “\textit{lasciare la panchina e scendere in campo}” facendomi capire che, invece di continuare ad arrampicarmi, avrei dovuto imparare a nuotare. Senza di voi, alcune scalate sarebbero state davvero troppo difficili da affrontare.

\noindent \textit{Per aspera ad astra!}

\begin{flushright}
    \textbf{\textit{Antonio}}
\end{flushright}

\end{document}
